\begin{enumerate}
\item Cheney's algorithm
\label{sec:org9d148aa}

\begin{verbatim}
// LispObject is just a pointer type
typedef uintptr_t LispObject;

// LispObject is a pointer type
uintptr_t fringe1, limit1; // heap1, where all allocations happen 
uintptr_t fringe2, limit2; // heap2 used for copying GC

LispObject allocate(size_t size) {
  if (fringe1 + size > limit1) {
    collect();
  }

  if (fringe1 + size > limit1) {
     // We are out of memory. Try to increase memory
     // ...
  }

  uintptr_t result = fringe1;
  fringe1 += size;
  return result;
}

// Two helper functions are needed
LispObject copy(LispObject obj) {
  size_t len = size(obj);

  LispObject new_obj = static_cast<LispObject>(fringe2);
  fringe2 += len;
  return new_obj;
}

uintptr_t copycontent(LispObject obj) {
  for (auto ref&: forward_references(obj)) {
    ref = copy(ref);
  }

  return static_cast<uintptr_t>(obj) + size(obj);
}

void collect() {
  // First we copy over the root set, which includes symbols.
  for (LispObject& symbol: symbols_table) {
    symbol = copy(symbol);
  }

  uintptr_t s = heap2;
  while (fringe2 < fringe2) {
    s = copycontent(static_cast<LispObject>(s));
  }

  swap(fringe1, fringe2);
  swap(limit1, limit2);
}
\end{verbatim}

\item \texttt{Gc\_guard} and \texttt{Gc\_lock}
\label{sec:orgc9a1d3d}
\item Lock free symbol lookup
\label{sec:org0a6de58}
\item Thread pool
\label{sec:org504fab0}
\end{enumerate}