I have successfully implemented a parallel programming language. The ParVSL language
allows the user to employ multi-threading to speed up their algorithms. It offers
a simple shared memory model, based on mutual exclusion and condition variables,
without compromising on any features of the original language VSL. I have demonstrated
this by using it to build a large software project: the REDUCE Algebra System.
I then used it to implement a thread pool and tested it on parallel algorithms with
inter-thread communication, proving a large performance gain can be obtained.

Ultimately, I have shown that a Computer Algebra System can benefit from parallelism,
and have proved that it is possible to modify REDUCE, a large real-world application,
to employ multi-threading effectively. Developers of REDUCE are now able to
use ParVSL to test parallel versions of their algorithms, and some of these developers
have already started testing my system on their workflows.

\section{Lessons learned}

\subsection{Debugging multi-threaded code}

Retrofitting multi-threading into a system designed without parallelism in mind is bound to bring issues.
Many assumptions in original code were violated when running multiple threads. To find and fix all the issues
I had to thoroughly understand every part of the system, which took time.
Sometimes my changes exposed an interaction with some area of code
I had not yet analysed, causing unexpected bugs.

Multi-threaded code is non-deterministic
by nature so many issues would only show up in testing sometimes
when the threads would conflict in just the
right way. Bugs were not always reproducible,
meaning sometimes I only discovered some weeks after producing them,
when tests which had been passing suddenly failed.

Having a large system such as REDUCE provided a great set of regression
tests. However, oftentimes errors only showed up after running thousand of lines
of code. This was also code in a language (RLISP) running on top
of a language (LISP) running on top of a third language (C++), so the call stacks
generated were generally unhelpful.

\subsection{Cross platform performance}

Historically, one of the biggest difficulties of implementing multi-threaded languages
was providing cross platform support. I have tested ParVSL on the major platforms and
showed that is capable of supporting them. However, I also discovered that performance
across platforms is very inconsistent. System specific mechanisms of thread local storage
and mutual exclusion have very different implementations and characteristics. This meant
I could not replicate the performance achieved on Linux on other platforms. As of today,
an understanding of each system and careful programming of individual scenarios is still
necessary.

\subsection{Parallelism is hindered by imperative programming}

I discovered that the REDUCE Lisp language included a few historical design decisions which
limit the potential of parallelism, namely its side-effectful nature. The functional programming
paradigm makes it much easier to write safe high-performance multi-threaded programs by avoiding
data races. When using an imperative style, the language has to do extra work to maintain safety,
such as using mutual exclusion on variable access, which can severely slow down any algorithm.

RLISP already offers many features for functional programming.
Modifying REDUCE to avoid side-effectful features of the language, and using a more
functional approach (e.g. avoiding the use of global symbols or the modification
of lists in place) would be needed for it to make good use of multi-threading.

\section{Future work}

While ParVSL is able to support all of REDUCE, it is an interpreted language
which limits its performance. A large number of optimisations are unavailable in this
case and many checks are performed real-time, rather than statically, slowing it down.
The next step would be to use the lessons learned to modify VSL's brother: CSL.
CSL is compiled and faster than VSL, however it has a much larger code base.
While it would require more work, most of the lessons learned from writing ParVSL could be translated
to create a similar ParCSL.

The maintainers of REDUCE packages are now able to implement parallel algorithms
for computer algebra, as I have demonstrated with my implementation of
a parallel Buchberger's algorithm for computing Gröbner bases. I showed that
we can achieve speed-ups of over 100\% and it would be interesting to take
this further and evaluate other multi-threaded algorithms described in literature.

While I have focused on the implementation of the Lisp system, I did not formally define
the semantics of the new features. I talked about the trade-offs between safety and performance,
and these need to be carefully considered when establishing the set of guarantees to
provide for multi-threaded programs.