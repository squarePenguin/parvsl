I have successfully implemented a parallel programming language. The language ParVSL
allows the user the use multi-threading to speed up their algorithms. It offers
a simple shared memory model, based on mutual exclusion and condition variable,
without compromising on any features of the original language VSL. I have demonstrated
this by using it to build a large software project: the REDUCE Algebra System.
I then used to implement a thread pool and tested it on parallel algorithms with
inter-thread communication, proving a large performance gain can be obtained.

Ultimately, I have shown that a Computer Algebra System can benefit from parallelism,
and have proved that it is possible to modify REDUCE, a large real-world application
to use effectively employ multi-threading.

\section{Cross platform performance}

Historically, one of the biggest difficulty of implementing multi-threaded languages
was providing cross platform support. I have tested ParVSL on the major platforms and
showed that is capable of supporting them. However, I also discovered that performance
across platforms is very inconsistent. System specific mechanisms of thread local storage
and mutual exclusion have very different implementations and characteristics. This meant
I could not replicate the performance achieved on Linux on other platforms. As of today,
an understanding of each system and careful programming of individual scenarios is still
necessary.

\section{Parallelism is hindered by imperative programming}

I discovered that the REDUCE Lisp language included a few historical design decisions which
limit the potential of parallelism, namely its side-effectful nature. The functional programming
paradigm makes it much easier to write safe high-performance multi-threaded by avoiding
data races. When using an imperative style, the language has to do extra work to maintain safety,
such as using mutual exclusion on variable access, which can severely slow down any algorithm.
memory model which allows side-effects it is much more difficult

RLisp already offers many features for funtional programming.
Modifying REDUCE to avoid side-effectful features of the language, and using it in a more
functional would be needed for it to make good use of multi-threading.

\section{Future work}

While ParVSL was fully able to support all of the REDUCE, it is an interpreted language
which limits its performance. A large number of optimisations such is unavailable in this
case and many checks are performed real-time, rather than statically, slowing it down.
The next step would be to use use the lessons learned to modify VSL's compiled brother: CSL.
CSL is compiled and already much faster than VSL, however it has a much larger code base.
While it would take more work, most of the lessons from writing ParVSL could be translated
to ParCSL.