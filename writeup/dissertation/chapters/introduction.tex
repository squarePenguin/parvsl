The motivation for this project is to explore the implementation of multi-threading
capabilities within a working compiler and assess the benefits and trade-offs it brings
to a real-world application with a large, actively-developed body of code.

\section{REDUCE}

REDUCE \cite{reduce} is a portable general-purpose Computer Algebra System (CAS). It enables symbolic
manipulation of mathematical expressions and provides a wide range of algorithms
to solve problems spanning many fields.
It has a friendly user interface and can display maths and generate graphics.

REDUCE is one of a few open-source general-purpose CAS programs, alongside Maxima and Axiom.
The three projects are all built on top of different Lisp kernels. At the time of this writing,
none of these projects have any multi-threading capabilities. My aim is to remove this limitation
for REDUCE. The project is written in a language called RLisp, which is a specialised
Lisp dialect with an Algol-like syntax.

\section{VSL}

There are multiple implementations of the Lisp backend REDUCE uses: PSL, CSL and VSL.
Visual Standard Lisp (VSL) is an interpreted implementation written
in \texttt{C}. It is capable of building the entirety of REDUCE, supports all the major
platforms and architectures, and despite being slower than CSL or PSL it is capable of
running realistic calculations including the full REDUCE test suite.
It exists to provide a test-bed for ideas that may later move to the much larger,
compiled CSL version.

\section{Benefits of multithreading}
The idea of using parallel computing to speed up computer algebra has come
up in research papers for many years \cite{algebra-parallelism},
but much of the activity
pre-dates the now ubiquitous multi-core CPUs used in modern computers and the amount of memory
which they now provide. Moreover, advancements in single-core CPU performance have slowed down
significantly, as clock speeds have stagnated and even gone down in recent years. The biggest
area of improvement in these new CPUs is their core count and number of hardware threads.
Therefore, binding the performance of REDUCE to single-threaded performance limits the speed gains
brought by new hardware. This project involves building the infrastructure
REDUCE needs to take full advantage of today's hardware.

\section{Code examples}

To help explain the concepts that I introduce, code fragments will be used to show the algorithms.
These fragments will be a simplification of the original code, in order to remove the need
for context within the rest of the codebase. To that end, the names and function interfaces
are different from the real implementation. More complete versions of the code fragments
I showcase can be found in Appendix \ref{ch:appendix-code}.

For examples relating to the implementation of the interpreter, I will be using C++.
For work related to my evaluation, I will be showcasing the ParVSL language, which is a
Lisp variant. I also used an additional language running on top of Lisp, called RLisp,
in order to write code for REDUCE.
When the syntax of the language gets in the way of readability,
pseudo-code is used to aid the explanations.

\section{Achievements}
\label{sec:achievements}

My project involved modifications to several thousand lines of C/C++ code, upgrading it to
make better idiomatic use of C++ as well as adding concurrency support. It has also used raw
Lisp code in bracket notation, along with the rebuilding of REDUCE, which starts off as Lisp,
but is almost entirely written in its own language. REDUCE as a whole is around half a million
lines of code and my system fully supports it. I have also coded demonstration programs to
show how it can take advantage of modern multi-core processors to improve performance,
given the thread support I now provide it with. The existing set of REDUCE test scripts have provided me
with significant examples to demonstrate correctness and analyse the overheads of
concurrency support on all the major platforms (Linux, Windows and MacOS), with some surprising
differences between them.