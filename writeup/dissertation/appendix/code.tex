\section{Cheney's algorithm}
\label{sec:cheneycode}

\begin{verbatim}
// LispObject is just a pointer type
typedef uintptr_t LispObject;

// LispObject is a pointer type
uintptr_t fringe1, limit1; // heap1, where all allocations happen
uintptr_t fringe2, limit2; // heap2 used for copying GC

LispObject allocate(size_t size) {
  if (fringe1 + size > limit1) {
    collect();
  }

  if (fringe1 + size > limit1) {
     // We are out of memory. Try to increase memory
     // ...
  }

  uintptr_t result = fringe1;
  fringe1 += size;
  return result;
}

// Two helper functions are needed
LispObject copy(LispObject obj) {
  size_t len = size(obj);

  LispObject new_obj = static_cast<LispObject>(fringe2);
  fringe2 += len;
  return new_obj;
}

uintptr_t copycontent(LispObject obj) {
  for (auto ref&: forward_references(obj)) {
    ref = copy(ref);
  }

  return static_cast<uintptr_t>(obj) + size(obj);
}

void collect() {
  // First we copy over the root set, which includes symbols.
  for (LispObject& symbol: symbols_table) {
    symbol = copy(symbol);
  }

  uintptr_t s = heap2;
  while (fringe2 < fringe2) {
    s = copycontent(static_cast<LispObject>(s));
  }

  swap(fringe1, fringe2);
  swap(limit1, limit2);
}
\end{verbatim}

\section{\texttt{Gc\_guard} and \texttt{Gc\_lock}}
\label{sec:gclock-code}

\begin{verbatim}

\end{verbatim}

\section{Lock free symbol lookup}
\label{sec:lockfree-code}
\section{Thread pool}
\label{sec:threadpool-code}

\begin{verbatim}
  lisp;

  symbolic procedure queue;
      {nil, nil};

  symbolic procedure q_push(q, x);
      begin scalar back, newback;
          back := second q;
          newback := {x};

          if null back then <<
              rplaca(q, newback);
              rplaca(cdr q, newback); >>
          else <<
              rplacd(back, newback);
              rplaca(cdr q, newback); >>;
          return q;
      end;

  symbolic procedure q_pop(q);
      begin scalar front, next;
          front := first q;
          if null front then
              return {}
          else <<
              next := cdr front;
              if null next then rplaca(cdr q, {});
              rplaca(q, next);
              return (first front); >>;
      end;

  symbolic procedure q_empty(q);
      null (first q);

  symbolic procedure atomic(val);
      {mutex(), val};

  symbolic procedure atomic_set(a, val);
  begin
      scalar m;
      m := first a;

      mutexlock m;
      rplaca(cdr a, val);
      mutexunlock m;
  end;

  symbolic procedure atomic_get(a);
  begin
      scalar m, res;
      m := first a;

      mutexlock m;
      res := cadr a;
      mutexunlock m;
      return res;
  end;

  symbolic procedure safeq();
      {queue(), mutex(), condvar()};

  symbolic procedure safeq_push(sq, x);
      begin scalar q, m, cv;
          q := first sq;
          m := second sq;
          cv := third sq;

          % print "safeq push getting mutex";
          mutexlock m;
          % print "safeq push got mutex";
          q_push(q, x);
          condvar_notify_one cv;
          % print "safeq push unlocking mutex";
          mutexunlock m;
          return sq;
      end;

  symbolic procedure safeq_pop(sq);
      begin scalar q, m, cv, res;
          q := first sq;
          m := second sq;
          cv := third sq;
          res := nil;

          % print "safeq pop getting mutex";
          mutexlock m;
          % print "safeq pop got mutex";
          % print thread_id ();
          while q_empty q do condvar_wait(cv, m);
          res := q_pop q;
          % print "safeq pop unlocking mutex";
          mutexunlock m;
          % print "safeq pop done";
          return res;
      end;

  % non-blocking call
  symbolic procedure safeq_trypop(sq);
      begin scalar q, m, cv, res;
          q := first sq;
          m := second sq;
          cv := third sq;
          res := nil;

          % print "safeq trypop getting mutex";
          mutexlock m;
          % print "safeq trypop got mutex";

          if q_empty q then
              res := nil
          else
              res := {q_pop q};

          % print "safeq trypop unlocking mutex";
          mutexunlock m;
          return res;
      end;

  symbolic procedure safeq_empty(sq);
      begin scalar r, m;
          m := second sq;
          % print "safeq empty getting mutex";
          mutexlock m;
          % print "safeq empty got mutex";
          r := q_empty (first sq);
          mutexunlock m;
          return r;
      end;

  symbolic procedure future();
      {mutex (), nil};

  % blocking call to wait for future result
  symbolic procedure future_get(fut);
  begin
      scalar m, state, cv, res;
      m := first fut;
      % print "future get getting mutex";
      mutexlock m;
      % print "future get got mutex";

      state := second fut;

      if state = 'done then <<
          res := third fut;
          mutexunlock m;
          return res >>;

      if state = 'waiting then
          cv := third fut
      else <<
          cv := condvar ();
          rplacd(fut, {'waiting, cv}) >>;

      % print "future waiting cv";
      condvar_wait(cv, m);
      % print "future got signaled cv";
      % ASSERT: promise is fulfilled here

      res := third fut;
      % print "future get unlocking mutex";
      mutexunlock m;

      return res;
  end;

  % non-blocking call for future result
  % can wait on cv until timeout
  symbolic procedure future_tryget(fut, timeout);
  begin
      scalar m, state, cv, res;
      m := first fut;
      % print "future tryget getting mutex";
      mutexlock m;
      % print "future tryget got mutex";

      state := second fut;

      if state = 'done then
          res := {third fut}
      else if timeout = 0 then
          res := nil
      else <<
          if state = 'waiting then
              cv := third fut
          else <<
              cv := condvar ();
              rplacd(fut, {'waiting, cv}) >>;

          if condvar_wait_for(cv, m, timeout) then
              res := {third fut}
          else
              res := nil >>;

      % print "future tryget unlocking mutex";
      mutexunlock m;

      return res;
  end;

  symbolic procedure future_set(fut, value);
  begin
      scalar m, state;
      m := first fut;

      % print "future set getting mutex";
      mutexlock m;
      % print "future set got mutex";
      state := second fut;

      if state = 'done then
          error("future already set");

      if state = 'waiting then
          condvar_notify_all third fut;

      rplacd(fut, {'done, value});

      % print "future set unlocking mutex";
      mutexunlock m;
  end;

  symbolic procedure tp_runjob(tp);
  begin
      scalar tp_q, job, resfut, f, args, res;
      tp_q := first tp;
      job := safeq_trypop tp_q;
      if null job then thread_yield ()
      else <<
          job := first job;
          resfut := first job;
          f := second job;
          args := third job;
          res := errorset({'apply, mkquote f, mkquote args}, t);
          future_set(resfut, res);
          % print "done job"
      >>
  end;

  symbolic procedure thread_pool_job(tp_q, status);
  begin
      scalar job, resfut, f, args, res, stat;
      % print "Started worker";
      job := safeq_trypop tp_q;
      repeat <<
          if job then <<
              % print "got job";
              job := first job;
              resfut := first job;
              f := second job;
              args := third job;
              % res := apply(f, args);
              res := errorset({'apply, mkquote f, mkquote args}, t);
              future_set(resfut, res);
              % print "done job";
          >> else <<
              % print "yielding";
              thread_yield ();
          >>;
          job := safeq_trypop tp_q;
          stat := atomic_get status;
      >> until (stat = 'kill) or (stat = 'stop and null job);
      % print "shutting down thread_pool worker";

      return nil
  end;

  symbolic procedure thread_pool(numthreads);
      begin scalar tp_q, status, threads;
          tp_q := safeq();
          status := atomic 'run;
          threads := {};
          % print "starting workers";
          for i := 1:numthreads do threads := thread2('thread_pool_job, {tp_q, status}) . threads;
          return {tp_q, status, threads};
      end;

  symbolic procedure tp_addjob(tp, f, args);
  begin
      scalar tp_q, status, resfut;
      tp_q := first tp;
      status := atomic_get (second tp);

      if not (status = 'run) then
          return nil
      else <<
          resfut := future ();
          % print "pushing job";
          safeq_push(tp_q, {resfut, f, args});
          return resfut;
      >>;
  end;

  symbolic procedure tp_stop(tp);
  begin
      scalar threads;
      % print "tp_stop";
      atomic_set(second tp, 'stop);
      % print "atomic set";
      threads := third tp;
      for each td in threads do <<
          % print "joining thread"; print td;
          jointhread td;
          % print "joined thread";
      >>;
      % print "tp stopped";
      return nil;
  end;

  symbolic procedure tp_kill(tp);
  begin
      scalar threads;
      atomic_set(second tp, 'kill);
      threads := third tp;
      for each td in threads do <<
          % print "joining thread"; print td;
          jointhread td;
          % print "joined thread";
      >>;
  end;

  % tp := thread_pool();

  end;
\end{verbatim}