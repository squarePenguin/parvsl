\section{Cheney's algorithm}
\label{sec:cheneycode}

\begin{code}
\begin{minted}[breaklines,mathescape]{c++}
// LispObject is just a pointer type
typedef uintptr_t LispObject;

// LispObject is a pointer type
uintptr_t fringe1, limit1; // heap1, where all allocations happen
uintptr_t fringe2, limit2; // heap2 used for copying GC

LispObject allocate(size_t size) {
  // skip over pinned items
  if (is_pinned(fringe1)) {
    fringe1 += sizeof(LispObject);
  }

  if (fringe1 + size > limit1) {
    collect();
  }

  if (fringe1 + size > limit1) {
     // We are out of memory. Try to increase memory
     // ...
  }

  uintptr_t result = fringe1;
  fringe1 += size;
  return result;
}

// Two helper functions are needed
LispObject copy(LispObject obj) {
  size_t len = size(obj);

  LispObject new_obj = static_cast<LispObject>(fringe2);

  if (is_pinned(fringe2)) {
    // skip over pinned items;
    fringe2 += sizeof(LispObject);
  }

  fringe2 += len;
  return new_obj;
}

uintptr_t copycontent(LispObject obj) {
  for (auto& ref: references(obj)) {
    ref = copy(ref);
  }

  return static_cast<uintptr_t>(obj) + size(obj);
}

void collect() {
  c_stack_head = approximate_stack_pointer();

  // scan the stack from its head to base
  for (uintptr_t s  = c_stack_head;
                 s  < c_stack_base;
                 s  += sizeof(LispObject))
  if (in_heap(s)) { // check if s points to the virtual heap
    set_pinned(s); // found an ambiguous root
  }

  // remember the starting point of copied locations
  uintptr_t start_fringe2 = fringe2;

  // First we copy over the root set, which includes symbols.
  for (LispObject& symbol: symbols_table) {
    symbol = copy(symbol);
  }
  ...

  uintptr_t s = start_fringe2;
  while (s < fringe2);
    s = copycontent(static_cast<LispObject>(s));
  }

  swap(fringe1, fringe2);
  swap(limit1, limit2);
}
\end{minted}
\end{code}

\section{\texttt{Gc\_guard} and \texttt{Gc\_lock}}
\label{sec:gclock-code}

\begin{code}
\begin{minted}[breaklines,mathescape]{c++}
namespace {
std::atomic_int num_threads(0);
std::atomic_int paused_threads(0);
std::condition_variable gc_waitall;
std::condition_variable gc_cv;
std::atomic_bool gc_on(false);

std::mutex gc_guard_mutex;
std::mutex gc_lock_mutex;
}

class Gc_guard {
private:
  // Global mutex shared by Gc_guard instance.
  static std::mutex gc_guard_mutex;

public:
  Gc_guard() {
    td.c_stack_head = approximate_stack_pointer();

    // This thread is now safe for GC.
    safe_threads += 1;

    // Notify the thread waiting for garbage collection.
    gc_wait_cv.notify_one();
  }

  ~Gc_guard() {
    std::unique_lock<std::mutex> lock(gc_guard_mutex);

    // Wait until GC is done.
    gc_done_cv.wait(lock, []() { return !gc_on; });

    // Thread is not longer safe for GC.
    safe_threads -= 1;

    // Thread stack head will be invalidated when execution resumes.
    thread_data.c_stack_head = nullptr;
  }
};

class Gc_lock {
private:
  // Global lock shared by Gc_lock instances.
  static std::mutex gc_lock_mutex;

  // Prevents other threads from acquiring the Gc_lock.
  std::unique_lock<std::mutex> lock;

  // The GC thread needs to be in a safe state as well.
  Gc_guard gc_guard;

public:
  // Initialise the lock and guard before Gc_lock is constructed.
  Gc_lock() : lock(gc_lock_mutex), gc_guard() {
    gc_on = true;

    // Wait until all threads are in a safe state.
    gc_wait_cv.wait(lock, []() {
      return paused_threads == num_threads;
    });
  }

  ~Gc_lock() {
    gc_on = false;

    // Notify all threads that GC is over.
    gc_done_cv.notify_all();
  }
};
\end{minted}
\end{code}

\section{Shallow binding}
\label{sec:shallowbind-code}

\begin{code}
\begin{minted}[breaklines,mathescape]{c++}
class Shallow_bind {
private:
  int loc;
  LispObject save;
public:
  Shallow_bind(LispObject x, LispObject tval) {
    if (is_global(x)) {
      error1("shallow bind global", qpname(x));
    }

    loc = qfixnum(qvalue(x));
    LispObject& sv = td.local_symbol(loc);
    save = sv;
    sv = tval;
  }

  Shallow_bind(Shallow_bind&&) noexcept = default;

  ~Shallow_bind() {
    td.local_symbol(loc) = save;
  }
};
\end{minted}
\end{code}

\section{Lock free symbol lookup}
\label{sec:lockfree-code}

\begin{code}
\begin{minted}[breaklines,mathescape]{c++}
/**
* [search_bucket] searches a particular bucket in the symbol table
* It can search the whole bucket, or down to a location.
* If specifying [stop] make sure it is in the bucket, otherwise it will loop.
* Returns -1 if not found.
* */
LispObject search_bucket(LispObject bucket, const char *name, size_t len, LispObject stop=tagFIXNUM) {
    for (LispObject w = bucket; w != stop; w = qcdr(w)) {
        LispObject a = qcar(w);    // Will be a symbol.
        LispObject n = qpname(a);      // Will be a string.
        size_t l = veclength(qheader(n)); // Length of the name.

        if (l == len && strncmp(name, qstring(n), len) == 0) {
            return a;                  // Existing symbol found.
        }
    }

    return -1;
}

LispObject lookup(const char *name, size_t len, int flag)
{
    size_t loc = 1;
    for (size_t i = 0; i < len; i += 1) loc = 13 * loc + name[i];
    loc = loc % OBHASH_SIZE;

    LispObject bucket = obhash[loc].load(std::memory_order_acquire);
    LispObject s = search_bucket(bucket, name, len);

    if (s != -1) return s; // found the symbol

    if ((flag & 1) == 0) return undefined;
    LispObject pn = makestring(name, len);
    LispObject sym = allocatesymbol(pn);

    LispObject new_bucket = cons(sym, bucket);

    while (!obhash[loc].compare_exchange_strong(bucket, new_bucket, std::memory_order_acq_rel)) {
        LispObject old_bucket = bucket;
        bucket = obhash[loc].load(std::memory_order_acquire);

        // Reaching here means bucket is constructed from old_bucket
        s = search_bucket(bucket, name, old_bucket);

        if (s != -1) return s; // another thread has created the symbol in the meantime

        new_bucket = cons(sym, bucket);
    }

    // successfully inserted the symbol in the hash, can return it.
    return sym;
}
\end{minted}
\end{code}

\section{Thread-local symbols}
\label{sec:threadlocal-code}

\begin{code}
\begin{minted}[breaklines,mathescape]{c++}
std::atomic_int num_symbols(0);

thread_local std::vector<LispObject> fluid_locals;
std::vector<LispObject> fluid_globals; // the global values

/**
 * Returns a shared id to index the symbol.
 * Used internally to identify the same symbol name on multiple
 * threads. This location will be an index into the actual storage array,
 * whether it's global or thread_local, and will be stored inside
 * qvalue(x).
 * */
int allocate_symbol() {
  std::unique_lock<std::mutex> lock(alloc_symbol_mutex);
  int loc = num_symbols;
  num_symbols += 1;
  fluid_globals.resize(num_symbols, undefined);
  return loc;
}

/**
* [local_symbol] gets the thread local symbol. may return undefined.
* Note, thread_local symbols are only lazily resised. Accessing a local
* symbol directly is dangerous. You need to use this function to ensure the
* local symbol is at least allocated.
*/
LispObject& local_symbol(int loc) {
  if (num_symbols > (int)fluid_locals.size()) {
    fluid_locals.resize(num_symbols, undefined);
  }

  return fluid_locals[loc];
}

/**
* [symval] returns the real current value of the symbol on the current thread.
* It handles, globals, fluid globals, fluid locals and locals.
* should be used to get the true symbol, instead of qvalue(s).
*/
LispObject& par_value(LispObject s) {
  if (is_global(s)) {
    return qvalue(s);
  }

  int loc = qfixnum(qvalue(s));
  LispObject& res = td.local_symbol(loc);
  // Here I assume undefined is a sort of "reserved value", meaning it can only exist
  // when the object is not shallow_bound. This helps me distinguish between fluids that
  // are actually global and those that have been bound.
  // When the local value is undefined, I refer to the global value.
  if (is_fluid(s) && res == undefined) {
    return fluid_globals[loc];
  }
  // This is either local or locally bound fluid
  return res;
}
\end{minted}
\end{code}

\section{Thread pool}
\label{sec:threadpool-code}

\begin{code}
\begin{minted}{text}
symbolic procedure atomic(val);
  {mutex(), val};

symbolic procedure atomic_set(a, val);
begin
  scalar m := first a;
  mutexlock m;
  rplaca(cdr a, val);
  mutexunlock m;
end;

symbolic procedure atomic_get(a);
begin
  scalar res, m := first a;
  mutexlock m;
  res := cadr a;
  mutexunlock m;
  return res;
end;

symbolic procedure safeq();
  {queue(), mutex(), condvar()};

symbolic procedure safeq_push(sq, x);
begin scalar q, m, cv;
  q := first sq; m := second sq; cv := third sq;
  mutexlock m;
  q_push(q, x);
  condvar_notify_one cv;
  mutexunlock m;
  return sq;
end;

symbolic procedure safeq_pop(sq);
begin scalar q, m, cv, res;
  q := first sq; m := second sq; cv := third sq; res := nil;
  mutexlock m;
  while q_empty q do condvar_wait(cv, m);
  res := q_pop q;
  mutexunlock m;
  return res;
end;

% non-blocking call
symbolic procedure safeq_trypop(sq);
begin scalar q, m, cv, res;
  q := first sq; m := second sq; cv := third sq; res := nil;
  mutexlock m;
  if q_empty q then res := nil
  else res := {q_pop q};
  mutexunlock m;
  return res;
end;

symbolic procedure safeq_empty(sq);
begin scalar r, m;
  m := second sq;
  mutexlock m;
  r := q_empty (first sq);
  mutexunlock m;
  return r;
end;

symbolic procedure future();
  {mutex (), nil};

% blocking call to wait for future result
symbolic procedure future_get(fut);
begin
  scalar m, state, cv, res;
  m := first fut;
  mutexlock m;
  state := second fut;
  if state = 'done then <<
    res := third fut;
    mutexunlock m;
    return res >>;

  if state = 'waiting then
    cv := third fut
  else <<
    cv := condvar ();
    rplacd(fut, {'waiting, cv}) >>;

  condvar_wait(cv, m);
  % ASSERT: promise is fulfilled here

  res := third fut;
  mutexunlock m;
  return res;
end;

% non-blocking call for future result
% can wait on cv until timeout
symbolic procedure future_tryget(fut, timeout);
begin
  scalar state, cv, res, m := first fut;
  mutexlock m;

  state := second fut;

  if state = 'done then res := {third fut}
  else if timeout = 0 then res := nil
  else <<
    if state = 'waiting then
      cv := third fut
    else <<
      cv := condvar ();
      rplacd(fut, {'waiting, cv}) >>;

    if condvar_wait_for(cv, m, timeout) then
      res := {third fut}
    else
      res := nil >>;

  mutexunlock m;
  return res;
end;

symbolic procedure future_set(fut, value);
begin
  scalar state, m := first fut;
  mutexlock m;
  state := second fut;

  if state = 'done then
    error("future already set");
  if state = 'waiting then
    condvar_notify_all third fut;

  rplacd(fut, {'done, value});
  mutexunlock m;
end;

symbolic procedure tp_runjob(tp);
begin
  scalar tp_q, job, resfut, f, args, res;
  tp_q := first tp;
  job := safeq_trypop tp_q;
  if null job then thread_yield ()
  else <<
    job := first job;
    resfut := first job;
    f := second job;
    args := third job;
    res := errorset({'apply, mkquote f, mkquote args}, t);
    future_set(resfut, res);
  >>
end;

symbolic procedure thread_pool_job(tp_q, status);
begin
  scalar job, resfut, f, args, res, stat;
  job := safeq_trypop tp_q;
  repeat <<
    if job then <<
      job := first job;
      resfut := first job;
      f := second job;
      args := third job;
      res := errorset({'apply, mkquote f, mkquote args}, t);
      future_set(resfut, res);
    >> else <<
      thread_yield ();
    >>;
    job := safeq_trypop tp_q;
    stat := atomic_get status;
  >> until (stat = 'kill) or (stat = 'stop and null job);

  return nil
end;

symbolic procedure thread_pool(numthreads);
begin scalar tp_q, status, threads;
  tp_q := safeq();
  status := atomic 'run;
  threads := {};
  for i := 1:numthreads do
    threads := thread2('thread_pool_job, {tp_q, status}) . threads;
  return {tp_q, status, threads};
end;

symbolic procedure tp_addjob(tp, f, args);
begin
  scalar tp_q, status, resfut;
  tp_q := first tp;
  status := atomic_get (second tp);

  if not (status = 'run) then
    return nil
  else <<
    resfut := future ();
    safeq_push(tp_q, {resfut, f, args});
    return resfut;
  >>;
end;

symbolic procedure tp_stop(tp);
begin
  scalar threads;
  atomic_set(second tp, 'stop);
  threads := third tp;
  for each td in threads do <<
    jointhread td;
  >>;
  return nil;
end;
\end{minted}
\end{code}

\section{Parallel Buchberger's algorithm}
\label{sec:buchberger}

\begin{code}
\begin{minted}{text}
numthreads := 8;

%% [groe_worker] takes
%% [pairs_ref] - ref to the set of pairs to process
%% [L_ref]  - ref to the current basis
%% [m]  - the mutex
%% [numdone_ref] - for sync between threads (acts as a semaphore)
symbolic procedure groeworker(pairs_ref, L_ref, m, numdone_ref);
begin
    scalar s, pairs, L, L1, numdone, done, workpairs;

    done := nil;

% using gotos for lack of continue and nested return.
START:
    mutexlock m;
    pairs := car pairs_ref;
    numdone := car numdone_ref;

    % No more work to be done, return.
    if null pairs then <<
        if null done then <<
          numdone := numdone + 1;
          rplaca (numdone_ref, numdone);
          done := t;
        >>;

        if numdone = numthreads then <<
            % we are done
            mutexunlock m;
            goto FINISH;
        >> else <<
            mutexunlock m;

            % Allow other threads a chance
            thread_yield();
            goto START;
        >>;
    >> else if done then <<
        % Work queue no longer empty.
        numdone := numdone - 1;
        rplaca (numdone_ref, numdone);
        done := nil;
    >>;

    p := car pairs;
    rplaca (pairs_ref, cdr pairs);
    L := car L_ref;

    mutexunlock m;

    % Now I have the work that I perform for each task.
    % Create an s-polynomial.
    s := s_poly(car p, cdr p);

    % Reduce it be an unsynchronized current snapshot of the current draft base.
    s := reduce_by(s, L);

    mutexlock m;
    L := car L_ref; % unbox L again as it might have changed
    pairs := car pairs_ref;
    L1 := L;  % because I will update L as I scan it.

    % Now if any polynomial in the existing base would be divisible by the new
    % element I should remove it and all pending pairs using it. Doing this
    % should let me end up with a miminal basis.
    for each p in L1 do
        if not xless(dfx p, dfx s) then <<
            L := delete(p, L);
            pairs := delete_pairlist(p, pairs) >>;
    for each p in L do
        pairs := (s . p) . pairs;

    % pairs := randomise pairs;

    L := s . L;
    rplaca (L_ref, L); % box it back
    rplaca (pairs_ref, pairs);

    mutexunlock m;
    goto START;

FINISH:
    return nil;
end;

symbolic procedure babygroe L;
begin
    scalar pairs, pairs_ref, L_ref, m, threads, numdone_ref;
    L := for each p in L collect dfmake_monic p;
    pairs := for each p on L conc
      for each q in cdr p collect (car p . q);
    terpri();
    for each p in L do << dfprin p; terpri() >>;

    pairs := randomise pairs;
    pairs_tried := 0;
    basis_updates := 0;

    pairs_ref := {pairs};
    L_ref := {L};
    m := mutex ();
    numdone_ref := {0};

% Put all the pairs in the work queue so that the one at the start of
% the list "pairs" will be the first work packet to be run.
    threads := for i := 1:numthreads collect
        thread2('groeworker, {pairs_ref, L_ref, m, numdone_ref});

% Block here until all workers are done.
    for each thr in threads do jointhread thr;
    L := car L_ref;
    return L
end;
\end{minted}
\end{code}