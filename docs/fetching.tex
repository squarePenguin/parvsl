\chapter{Fetching, building and installing \vsl}
\section{Raspberry Pi and other Linux platforms}
The instructions here have been tested in a Raspberry Pi model B
with a freshly installed copy of Debian Linux from the version
{\tx debian6-19-04-2012}. Everything will fit on a 4Gbyte SD card but
it will be much more comfortable with one that is larger - but also
significantly more pain to perform the initial installation of the
operating system because of a need to re-size partitions so as to use
al available space. It seems best to assume that the various Raspberry Pi
starter guides and web-sites will provide careful blow-by-blow instructions
for operating system installation, and merely provide a brief overview
of the steps involved here. It is also probable that many users will
buy an SD card that already has Linux installed on it. If that is Debian
then you can skip the next few steps\ldots and that could be a real
relief!

\begin{description}
\item[Fetch {\tx debian6-19-04-2012.img}] It should be reasonable to fetch
any later version, most plausibly by checking the downloads area on the
Raspberry Pi web-site. Obviously if you use a newer version there could
in principle be slight differences in your experince, so check any
release notes or explanatory documentation associated with the version
you fetch. You may potentially save yourself serious grief later by
confirming that the image you downloaded has the correct check-sum. If you
do not  do that and end up with a corrupted download or a version where
somehow a few bytes have been damaged then the behaviour later may be
messy.
\item[Use {\tx win32diskmanager} to write the SD card] \ldots if you
used a Windows computer to fetch the debian image, or {\tx dd} if you happen
to have a Linux machine. On a Macintosh you may find
{\tx raspiwrite} does what you need. Detailed instructions should be
available via the Raspberry Pi web site.
\item[Re-size the partitions on the SD card] This so that you can benefit from
all the space on your SD card. Some Linux systems automatically arrange this
when you first start them up, but at least the initial Debian release for
Raspberry Pi does not. To re-size the partitions the smoothest scheme seems
to be to use {\tx gparted} from Linux -- and if you have a computer that
mostly runs Windows to boot it from a ``live'' Linux CD or DVD so you
can do this. Again check websites associated with the particular
operating system image that you download.
\item[Move the SD card to your Raspberry Pi and check it] It may at this
stage be necessary to reconfigure it to work with the particular monitor
you are using with it, or to reset it to start up with a graphical login
rather than merely a text one. 
\end{description}

You can perhaps see that obtaining a card that has been fully configured
and is 100\% ready for use is liable to make life a lot easier.

\section{Windows}
The instructions given here should apply for all versions of Windows
from XP up as far as Windows 8. It will be possible to use either
a 32-bit or a 64-bit version of the operating system without that having
any major effect on results. But please note that despite the name of
the operating system that is used this will be establishing a command-line
based environment in which you will work! You will need an internet connection
active while setting things up. Most of the steps that have to be taken
only need to be done once, and so will be presented here as
instructions to follow without huge amounts of explanation of the
background. But you can follow them through happy in the knowledge that the
software you are fetching is Free and Open Source so you will not have
to pay for it, the licences that apply to it are all friendly, and if you
become really keen and want to investigate everything in the most
minute detail that is not just possible and legal but is encouraged.

The first thing to do will be to launch a web browser and navigate to
\url{http://www.cygwin.com}. There you will find a link that lets you
fetch a utility names {\tx setup} that can install the ``cygwin''
tools and development environment. Download {\tx setup} (or more
pedantically {\tx setup.exe}) and store it in some folder. It could make
sense to save it as {\tx c:{\textbackslash}software{\textbackslash}cygwin-setup{\textbackslash}setup.exe}.

Now launch {\tx setup}. It should announce itself as the ``Cygwin Net
Release Setup Program'' and let you click on {\tx Next} to proceed.
Select ``Install from Internet'' then at the next stage you probably accept
the default behaviour and make the Cygwin Root Directory {\tx c:{\textbackslash}cygwin}
with the package installed for all users. At the next stage you can ask the
installer to use the directory where you saved {\tx setup} as your
Local Package Directory -- for instance this may be
{\tx c:{\textbackslash}software{\textbackslash}cygwin-setup}. There are options that need to be set if you
connect to the internet via some proxy. If Internet Explorer works for you
then you can copy and use the settings that it has.
Next you need to select a download site -- you should obviously scan the list
of alternatives you are given and try to select one that is liable to be
close to you. If your first choice gives trouble you can cancel the
installation and try another.

At this stage you are presented with a screen headed ``Select Packages'' where
you choose which components of cygwin to install. There is a search box towards
the top left. Type the implausible name {\tx mingw64-i686} into it and you
then need to expand the {\tx Devel} branch below by clicking on the
{\tx +} sign in a box. You should see a range of package names mostly
tagged as {\tx Skip}. If you click on the {\tx Skip} you can cycle round
selecting between the version numbers of any available versions, and
{\tx Keep}, {\tx Uninstall} and {\tx Reinstall} for packages you have already
installed. Select the highest available version for {\tx ming64-i686-gcc-core}.
If you are looking ahead to trying all the \vsl{} examples you should also
select {\tx mingw64-i686-gcc-g++}.

Next clear the search box and use it to find {\tx make} and select it, and then
{\tx subversion}. Each should be in the {\tx Devel} section.
\begin{marginpar}
{\em If you are a developer involved with the initial work on \vsl{}
you also need to fetch {\tx unifdef} and all of the packages
that are parts of {\tx texlive}. Well at least
{\tx texlive-collections-latexextra}. I will suggest
{\tx mingw64-i686-gcc-g++} and {\tx wget} too.}
\end{marginpar}
There are
obviously an enormous range of other modules you could select, but it is not
necessary to install all of them on your first try. You can just re-run
{\tx setup} at any later stage and get the ones you have already installed
updated and pick additional ones you find you might need. So click at the
bottom of the window to let {\tx setup} install everything for you. It will
download material (so just how long things take will depend on the speed
of your internet connection) and you will see a series of progress bars
as it installs everything. At the end it will offer to put an icon for
cygwin on your desktop and in the start menu, and it might be a good idea
to accept these suggestions.

If all has gone well you will end up with a cygwin icon on your desktop.
When you double-click it for the first time it does a little bit of
final setting up and then presents you with a window running a terminal.
There should (after a while) be a prompt visible there that ends in
``{\tx \$}''. The terminal window starts off by default showing white letters
on a black background in a rather small window. I generally like to
click on the icon at its top left and select {\tx options} so I
can change it to dark text on a light background using a font big enough that
it fills a reasonable proportion of my screen. To check that things are
working at all you might issue the command ``{\tx ls /usr/bin} ''. {\tx ls}
is the command that displays a list of the files present in a directory, and
{\tx /usr/bin} is where {\tx cygwin} keeps most of the commands it
makes available to you (including {\tx ls}!). If you see a reasonably long
list of cryptic names that includes {\tx ls} and {\tx make} all is well.
Note that many of the names may show up with a suffix {\tx .exe} that is part
of the full name that Windows knows them by -- this is expected and not
a problem.
If things do not work properly you may need to try {\tx setup} again.

There is one more once-only step you need to take. What you have done so far
installs an environment within which \vsl{} can be built and used. You now
need to fetch \vsl{} itself. So go\footnote{The source listed here will
NOT be the published one at all. As shown here it is just for
collaborators!}
{\small\begin{verbatim}
   svn co http://subversion.assembla.com/svn/rpistuff rpi
\end{verbatim}}
{\tx svn} is a utility whose full name is {\tx subversion} that can be
used by collaborating groups to manage projects that several people may
contribute to. Here it is being used to {\em Check out} (using the
abbreviation {\tx co}) the files associated with \vsl. You should see
a series of messages as individual files are fetched, and end up
so that if you go just {\tx ls} (without any specific location after the
command) you are shown that you now have a directory called {\tx rpi}.
You can navigate into this by saying {\tx cd~rpi/vsl} and now {\tx ls}
should show a collection of files including ones called
{\tx vsl.c} and {\tx Makefile}.

Now try to build \vsl{} by issuing the command
{\small\begin{verbatim}
   cd rpi/vsl
   make vsl
   make vsl.img
\end{verbatim}}
The output from the first line using {\tx make} should be similar to
{\small\begin{verbatim}
$ make vsl
i686-w64-mingw32-gcc -O2 -Wall vsl.c -lm -o vsl
\end{verbatim}}
with no obvious messages or complaints. The second use of {\tx make}
digests the \vsl{} library and saves it in a state where it can be available
by default in the future. Now as a quick test try:
{\small\begin{verbatim}
   ./vsl factorial.lsp
\end{verbatim}}
and if all has gone well you will see a table of factorials. If you manage
to reach this stage you can be confident that you have a system ready
to work with.

\section{Macintosh}


% @@@@@ incomplete



