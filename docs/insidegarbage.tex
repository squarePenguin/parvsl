\chapter{Storage management and Garbage Collection}


% @@@@@ incomplete

% Code to be explained...

{\small\begin{verbatim}
void allocateheap()
{   void *pool = (void *)
        malloc(ROUNDED_HEAPSIZE+BITMAPSIZE+ROUNDED_STACKSIZE+16);
    if (pool == NULL)
    {   printf("Not enough memory available: Unable to proceed\n");
        exit(1);
    }
    heap1base = (LispObject)pool;
    heap1base = (heap1base + 7) & ~7; // ensure alignment
    heap1top = heap2base = heap1base + (ROUNDED_HEAPSIZE/2);
    heap2top = heap2base + (ROUNDED_HEAPSIZE/2);
    fringe1 = heap1base;
    fpfringe1 = heap1top;
    fringe2 = heap2base;
    fpfringe2 = heap2top;
    stackbase = heap2top;
    stacktop = stackbase + ROUNDED_STACKSIZE;
    bitmap = stacktop;
}

extern void reclaim();

LispObject cons(LispObject a, LispObject b)
{
    if (fringe1 >= fpfringe1)
    {   push2(a, b);
        reclaim();
        pop2(b, a);
    }
    qcar(fringe1) = a;
    qcdr(fringe1) = b;
    a = fringe1;
    fringe1 += 2*sizeof(LispObject);
    return a;
}


#define swap(a,b) w = (a); (a) = (b); (b) = w;

extern LispObject copy(LispObject x);

int gccount = 1;

void reclaim()
{
// The strategy here is due to C J Cheyney ("A Nonrecursive List Compacting
// Algorithm". Communications of the ACM 13 (11): 677-678, 1970).
    LispObject *s, w;
    printf("+++ GC number %d", gccount++);
// I need to clear the part of the bitmap that could be relevant for floating
// point values.
    int o = (fpfringe1 - heap1base)/(8*8);
    while (o < BITMAPSIZE) ((unsigned char *)bitmap)[o++] = 0;
// Process everything that is on the stack.
    for (s=(LispObject *)stackbase; s<sp; s++) *s = copy(*s);
// I should also copy any other list base values here.
    for (o=0; o<BASES_SIZE; o++) bases[o] = copy(bases[o]);
    for (o=0; o<OBHASH_SIZE; o++)
        obhash[o] = copy(obhash[o]);
// Now perform the second part of Cheyney's algorithm, scanning the
// data that has been put in the new heap.
    s = (LispObject *)heap2base;
    while ((LispObject)s != fringe2)
    {   LispObject h = *s;
        if (!isHDR(h)) // The item to be processed is a simple cons cell
        {   *s++ = copy(h);
            *s = copy(*s);
            s++;
        }
        else              // The item is one that uses a header
            switch (h & TYPEBITS)
            {   case typeSYM:
                    w = ((LispObject)s) + tagSYMBOL;
                    // qflags(w) does not need adjusting
                    qvalue(w) = copy(qvalue(w));
                    qplist(w) = copy(qplist(w));
                    qpname(w) = copy(qpname(w));
                    // qdefn(w) does not need adjusting
                    qlits(w)  = copy(qlits(w));
                    s += 6;
                    continue;
                case typeSTRING:
                case typeBIGNUM:
// These only contain binary information, so none of their content needs
// any more processing.
                    w = (sizeof(LispObject) + veclength(h) + 7) & ~7;
                    s += w/sizeof(LispObject);
                    continue;
                case typeVEC:
                case typeEQHASH:
                case typeEQHASHX:
// These are to be processed the same way. They contain a bunch of
// reference items.
                    s++; // Past the header
                    w = veclength(h);
                    while (w > 0)
                    {   *s = copy(*s);
                        s++;
                        w -= sizeof(LispObject);
                    }
                    w = (LispObject)s;
                    w = (w + 7) & ~7;
                    s = (LispObject *)w;
                    continue;
                default:
                    // all the "spare" codes!
                    disaster(__LINE__);
            }
    }
// Finally flip the two heaps ready for next time.
    swap(heap1base, heap2base);
    swap(heap1top, heap2top);
    fringe1 = fringe2;
    fpfringe1 = fpfringe2;
    fringe2 = heap2base;
    fpfringe2 = heap2top;
    if (fpfringe1 - fringe1 < 1000*sizeof(LispObject))
    {   printf("\nRun out of memory.\n");
        exit(1);
    }
    printf(" - collection complete\n");
    fflush(stdout);
}

LispObject copy(LispObject x)
{   LispObject h;
    int o, b;
    switch (x & TAGBITS)
    {   case tagCONS:
            if (x == 0) disaster(__LINE__);
            h = *((LispObject *)x);
            if (isFORWARD(h)) return (h - tagFORWARD);
            qcar(fringe2) = h;
            qcdr(fringe2) = qcdr(x);
            h = fringe2;
            qcar(x) = tagFORWARD + h;
            fringe2 += 2*sizeof(LispObject);
            return h;
        case tagSYMBOL:
            h = *((LispObject *)(x - tagSYMBOL));
            if (isFORWARD(h)) return (h - tagFORWARD + tagSYMBOL);
            if (!isHDR(h)) disaster(__LINE__);
            h = fringe2 + tagSYMBOL;
            qflags(h) = qflags(x);
            qvalue(h) = qvalue(x);
            qplist(h) = qplist(x);
            qpname(h) = qpname(x);
            qdefn(h)  = qdefn(x);
            qlits(h)  = qlits(x);
            fringe2 += 6*sizeof(LispObject);
            qflags(x) = h - tagSYMBOL + tagFORWARD;
            return h;
        case tagATOM:
            if (x == NULLATOM) return x; // special case!
            h = qheader(x);
            if (isFORWARD(h)) return (h - tagFORWARD + tagATOM);
            if (!isHDR(h)) disaster(__LINE__);
            switch (h & TYPEBITS)
            {   case typeEQHASH:
// When a hash table is copied its header is changes to EQHASHX, which
// indicates that it will need rehashing before further use.
                    h ^= (typeEQHASH ^ typeEQHASHX);
                case typeEQHASHX:
                case typeSTRING:
                case typeVEC:
                case typeBIGNUM:
                    o = (int)veclength(h);  // number of bytes excluding the header
                    *((LispObject *)fringe2) = h; // copy header word across
                    h = fringe2 + tagATOM;
                    *((LispObject *)(x - tagATOM)) = fringe2 + tagFORWARD;
                    fringe2 += sizeof(LispObject);
                    x = x - tagATOM + sizeof(LispObject);
                    while (o > 0)
                    {   *((LispObject *)fringe2) = *((LispObject *)x);
                        fringe2 += sizeof(LispObject);
                        x += sizeof(LispObject);
                        o -= sizeof(LispObject);
                    }
                    fringe2 = (fringe2 + 7) & ~7;
                    return h;
                default:
                    //case typeSYM:
                    // also the spare codes!
                    disaster(__LINE__);
            }
        case tagFLOAT:
// every float is 8 bytes wide, regardless of what sort of machine I am on.
            h = (x - tagFLOAT - heap1base)/8;
            o = h/8;
            b = 1 << (h%8);
// now o is an offset and b a bit in the bitmap.
            if ((((unsigned char *)bitmap)[o] & b) != 0) // marked already.
                return *((LispObject *)(x-tagFLOAT));
            else
            {   ((unsigned char *)bitmap)[o] |= b; // mark it now.
                fpfringe2 -= sizeof(double);
                h = fpfringe2 + tagFLOAT;
                qfloat(h) = qfloat(x);             // copy the float.
                *((LispObject *)(x-tagFLOAT)) = h; // write in forwarding address.
                return h;
            }
        case tagFIXNUM:
            return x;
        default:
//case tagFORWARD:
//case tagHDR:
            disaster(__LINE__);
            return 0;  // avoid GCC moans.
    }
}

\end{verbatim}}
