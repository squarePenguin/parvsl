% @@@@@ incomplete

% Code that needs explaining...
{\small\begin{verbatim}

#define qfixnum(x)     ((x) >> 3)
#define packfixnum(n)  ((((LispObject)(n)) << 3) + tagFIXNUM)

#define MIN_FIXNUM     qfixnum(INTPTR_MIN)
#define MAX_FIXNUM     qfixnum(INTPTR_MAX)

#define qfloat(x)      (((double *)((x)-tagFLOAT))[0])

#define isBIGNUM(x) (isATOM(x) && ((qheader(x) & TYPEBITS) == typeBIGNUM))
#define qint64(x) (*(int64_t *)((x) - tagATOM + 8))


LispObject boxint64(int64_t a)
{   LispObject r = allocateatom(16);
    qheader(r) = tagHDR + typeBIGNUM + packlength(16);
    qint64(r) = a;
    return r;
}

// I will try to have a general macro that will help me with bringing
// everything to consistent numeric types - ie I can start off with a
// mix of fixnums, bignums and floats. The strategy here is that if either
// args is a float then the other is forced to that, and then for all sorts
// of pure integer work everything will be done as int64_t

#define NUMOP(name, a, b)                                                \
    if (isFLOAT(a))                                                      \
    {   if (isFLOAT(b)) return FF(qfloat(a), qfloat(b));                 \
        else if (isFIXNUM(b)) return FF(qfloat(a), (double)qfixnum(b));  \
        else if (isBIGNUM(b)) return FF(qfloat(a), (double)qint64(b));   \
        else return error1("Bad argument for " name, b);                 \
    }                                                                    \
    else if (isBIGNUM(a))                                                \
    {   if (isFLOAT(b)) return FF((double)qint64(a), qfloat(b));         \
        else if (isFIXNUM(b)) return BB(qint64(a), (int64_t)qfixnum(b)); \
        else if (isBIGNUM(b)) return BB(qint64(a), qint64(b));           \
        else return error1("Bad argument for " name, b);                 \
    }                                                                    \
    else if (isFIXNUM(a))                                                \
    {   if (isFLOAT(b)) return FF((double)qfixnum(a), qfloat(b));        \
        else if (isFIXNUM(b)) return BB((int64_t)qfixnum(a),             \
                                        (int64_t)qfixnum(b));            \
        else if (isBIGNUM(b)) return BB((int64_t)qfixnum(a), qint64(b)); \
        else return error1("Bad argument for " name, b);                 \
    }                                                                    \
    else return error1("Bad argument for " name, a)

#define UNARYOP(name, a)                                                 \
    if (isFIXNUM(a)) return BB((int64_t)qfixnum(a));                     \
    else if (isFLOAT(a)) return FF(qfloat(a));                           \
    else if (isBIGNUM(a)) return BB(qint64(a));                          \
    else return error1("Bad argument for " name, a)

// Similar, but only supporting integer (not floating point) values

#define INTOP(name, a, b)                                                \
    if (isBIGNUM(a))                                                     \
    {   if (isFIXNUM(b)) return BB(qint64(a), (int64_t)qfixnum(b));      \
        else if (isBIGNUM(b)) return BB(qint64(a), qint64(b));           \
        else return error1("Bad argument for " name, b);                 \
    }                                                                    \
    else if (isFIXNUM(a))                                                \
    {   if (isFIXNUM(b)) return BB((int64_t)qfixnum(a),                  \
                                   (int64_t)qfixnum(b));                 \
        else if (isBIGNUM(b)) return BB((int64_t)qfixnum(a), qint64(b)); \
        else return error1("Bad argument for " name, b);                 \
    }                                                                    \
    else return error1("Bad argument for " name, a)

#define UNARYINTOP(name, a)                                              \
    if (isFIXNUM(a)) return BB((int64_t)qfixnum(a));                     \
    else if (isBIGNUM(a)) return BB(qint64(a));                          \
    else return error1("Bad argument for " name, a)

// This takes an arbitrary 64-bit integer and returns either a fixnum
// or a bignum as necessary.

LispObject makeinteger(int64_t a)
{   if (a >= MIN_FIXNUM && a <= MAX_FIXNUM) return packfixnum(a);
    else return boxint64(a);
}

#undef FF
#undef BB
#define FF(a) boxfloat(-(a))
#define BB(a) makeinteger(-(a))

LispObject Nminus(LispObject a)
{   UNARYOP("minus", a);
}

#undef FF
#undef BB
#define FF(a, b) boxfloat((a) + (b))
#define BB(a, b) makeinteger((a) + (b))

LispObject Nplus2(LispObject a, LispObject b)
{   NUMOP("plus", a, b);
}

#undef FF
#undef BB
#define FF(a, b) boxfloat((a) * (b))
#define BB(a, b) makeinteger((a) * (b))

LispObject Ntimes2(LispObject a, LispObject b)
{   NUMOP("times", a, b);
}

#undef BB
#define BB(a, b) makeinteger((a) & (b))

LispObject Nlogand2(LispObject a, LispObject b)
{   INTOP("logand", a, b);
}

#undef BB
#define BB(a, b) makeinteger((a) | (b))

LispObject Nlogor2(LispObject a, LispObject b)
{   INTOP("logor", a, b);
}

#undef BB
#define BB(a, b) makeinteger((a) ^ (b))

LispObject Nlogxor2(LispObject a, LispObject b)
{   INTOP("logxor", a, b);
}

#undef FF
#undef BB


#define NARY(x, base, combinefn)      \
    LispObject r;                     \
    if (!isCONS(x)) return base;      \
    push(x);                          \
    r = eval(qcar(x));                \
    pop(x);                           \
    if (unwindflag != unwindNONE)     \
        return nil;                   \
    x = qcdr(x);                      \
    while (isCONS(x))                 \
    {   LispObject a;                 \
        push2(x, r);                  \
        a = eval(qcar(x));            \
        if (unwindflag != unwindNONE) \
        {   pop2(r, x);               \
            return nil;               \
        }                             \
        pop(r);                       \
        r = combinefn(r, a);          \
        pop(x);                       \
        x = qcdr(x);                  \
    }                                 \
    return r

LispObject Lplus(LispObject lits, LispObject x)
{   NARY(x, packfixnum(0), Nplus2);
}

LispObject Ltimes(LispObject lits, LispObject x)
{   NARY(x, packfixnum(1), Ntimes2);
}

LispObject Llogand(LispObject lits, LispObject x)
{   NARY(x, packfixnum(-1), Nlogand2);
}

LispObject Llogor(LispObject lits, LispObject x)
{   NARY(x, packfixnum(0), Nlogor2);
}

LispObject Llogxor(LispObject lits, LispObject x)
{   NARY(x, packfixnum(0), Nlogxor2);
}

LispObject Lnumberp(LispObject lits, int nargs, ...)
{   ARG1("numberp", x);
    return (isFIXNUM(x) || isBIGNUM(x) || isFLOAT(x) ? lisptrue : nil);
}

LispObject Lfixp(LispObject lits, int nargs, ...)
{   ARG1("fixp", x);
    return (isFIXNUM(x) || isBIGNUM(x) ? lisptrue : nil);
}

LispObject Lfloatp(LispObject lits, int nargs, ...)
{   ARG1("floatp", x);
    return (isFLOAT(x) ? lisptrue : nil);
}

LispObject Lfix(LispObject lits, int nargs, ...)
{   ARG1("fix", x);
    return (isFIXNUM(x) || isBIGNUM(x) ? x :
            isFLOAT(x) ? boxint64((int64_t)qfloat(x)) :
            error1("arg for fix", x));
}

LispObject Lfloor(LispObject lits, int nargs, ...)
{   ARG1("floor", x);
    return (isFIXNUM(x) || isBIGNUM(x) ? x :
            isFLOAT(x) ? boxint64((int64_t)floor(qfloat(x))) :
            error1("arg for floor", x));
}

LispObject Lceiling(LispObject lits, int nargs, ...)
{   ARG1("ceiling", x);
    return (isFIXNUM(x) || isBIGNUM(x) ? x :
            isFLOAT(x) ? boxint64((int64_t)ceil(qfloat(x))) :
            error1("arg for ceiling", x));
}

LispObject Lfloat(LispObject lits, int nargs, ...)
{   ARG1("float", x);
    return (isFLOAT(x) ? x :
            isFIXNUM(x) ? boxfloat((double)qfixnum(x)) :
            isBIGNUM(x) ? boxfloat((double)qint64(x)) :
            error1("arg for fix", x));
}

#define floatval(x)                   \
   isFLOAT(x) ? qfloat(x) :           \
   isFIXNUM(x) ? (double)qfixnum(x) : \
   isBIGNUM(x) ? (double)qint64(x) :  \
   0.0

LispObject Lcos(LispObject lits, int nargs, ...)
{   ARG1("cos", x);
    return boxfloat(cos(floatval(x)));
}

LispObject Lsin(LispObject lits, int nargs, ...)
{   ARG1("sin", x);
    return boxfloat(sin(floatval(x)));
}

LispObject Lsqrt(LispObject lits, int nargs, ...)
{   ARG1("sqrt", x);
    return boxfloat(sqrt(floatval(x)));
}

LispObject Llog(LispObject lits, int nargs, ...)
{   ARG1("log", x);
    return boxfloat(log(floatval(x)));
}

LispObject Lexp(LispObject lits, int nargs, ...)
{   ARG1("exp", x);
    return boxfloat(exp(floatval(x)));
}

LispObject Latan(LispObject lits, int nargs, ...)
{   ARG1("atan", x);
    return boxfloat(atan(floatval(x)));
}


// Now some numeric functions

#undef FF
#undef BB
#define FF(a, b) ((a) > (b) ? lisptrue : nil)
#define BB(a, b) ((a) > (b) ? lisptrue : nil)

LispObject Lgreaterp(LispObject lits, int nargs, ...)
{   ARG2("greaterp", x, y);
    NUMOP("greaterp", x, y);
}

#undef FF
#undef BB
#define FF(a, b) ((a) >= (b) ? lisptrue : nil)
#define BB(a, b) ((a) >= (b) ? lisptrue : nil)

LispObject Lgeq(LispObject lits, int nargs, ...)
{   ARG2("geq", x, y);
    NUMOP("geq", x, y);
}

#undef FF
#undef BB
#define FF(a, b) ((a) < (b) ? lisptrue : nil)
#define BB(a, b) ((a) < (b) ? lisptrue : nil)

LispObject Llessp(LispObject lits, int nargs, ...)
{   ARG2("lessp", x, y);
    NUMOP("lessp", x, y);
}

#undef FF
#undef BB
#define FF(a, b) ((a) <= (b) ? lisptrue : nil)
#define BB(a, b) ((a) <= (b) ? lisptrue : nil)

LispObject Lleq(LispObject lits, int nargs, ...)
{   ARG2("leq", x, y);
    NUMOP("leq", x, y);
}

LispObject Lminus(LispObject lits, int nargs, ...)
{   ARG1("minus", x);
    return Nminus(x);
}

LispObject Lminusp(LispObject lits, int nargs, ...)
{   ARG1("minusp", x);
// Anything non-numeric will not be negative!
    if ((isFIXNUM(x) && x < 0) ||
        (isFLOAT(x) && qfloat(x) < 0.0) ||
        (isATOM(x) &&
         (qheader(x) & TYPEBITS) == typeBIGNUM &&
         qint64(x) < 0)) return lisptrue;
    else return nil;
}

#undef BB
#define BB(a) makeinteger(~(a))

LispObject Llognot(LispObject lits, int nargs, ...)
{   ARG1("lognot", x);
    UNARYINTOP("lognot", x);
}

LispObject Lzerop(LispObject lits, int nargs, ...)
{   ARG1("zerop", x);
// Note that a bignum can never be zero! Because that is not "big".
// This code is generous and anything non-numeric is not zero.
    if (x == packfixnum(0) ||
        (isFLOAT(x) && qfloat(x) == 0.0)) return lisptrue;
    else return nil;
}

LispObject Lonep(LispObject lits, int nargs, ...)
{   ARG1("onep", x);
    if (x == packfixnum(1) ||
        (isFLOAT(x) && qfloat(x) == 1.0)) return lisptrue;
    else return nil;
}

#undef FF
#undef BB
#define FF(a) boxfloat((a) + 1.0)
#define BB(a) makeinteger((a) + 1)

LispObject Ladd1(LispObject lits, int nargs, ...)
{   ARG1("add1", x);
    UNARYOP("add1", x);
}

#undef FF
#undef BB
#define FF(a) boxfloat((a) - 1.0)
#define BB(a) makeinteger((a) - 1)

LispObject Lsub1(LispObject lits, int nargs, ...)
{   ARG1("sub1", x);
    UNARYOP("sub1", x);
}

#undef FF
#undef BB
#define FF(a, b) boxfloat((a) - (b))
#define BB(a, b) makeinteger((a) - (b))

LispObject Ldifference(LispObject lits, int nargs, ...)
{   ARG2("difference", x, y);
    NUMOP("difference", x, y);
}

#undef FF
#undef BB
#define FF(a, b) ((b) == 0.0 ? error1("division by 0.0", nil) : \
                  boxfloat((a) / (b)))
#define BB(a, b) ((b) == 0 ? error1("division by 0", nil) : \
                  makeinteger((a) / (b)))

LispObject Lquotient(LispObject lits, int nargs, ...)
{   ARG2("quotient", x, y);
    NUMOP("quotient", x, y);
}

#undef BB
#define BB(a, b) ((b) == 0 ? error1("remainder by 0", nil) : \
                  makeinteger((a) % (b)))

LispObject Lremainder(LispObject lits, int nargs, ...)
{   ARG2("remainder", x, y);
    INTOP("remainder", x, y);
}

#undef BB
#define BB(a, b) ((b) == 0 ? error1("division by 0", nil) : \
                  cons(makeinteger((a) / (b)), makeinteger((a) % (b))))

LispObject Ldivide(LispObject lits, int nargs, ...)
{   ARG2("divide", x, y);
    INTOP("divide", x, y);
}

#undef BB
#define BB(a) makeinteger((a) << sh)

LispObject Lleftshift(LispObject lits, int nargs, ...)
{   int sh;
    ARG2("leftshift", x, y);
    if (!isFIXNUM(y)) return error1("Bad argument for leftshift", y);
    sh = (int)qfixnum(y);
    UNARYINTOP("leftshift", x);
}

#undef BB
#define BB(a) makeinteger((a) >> sh)

LispObject Lrightshift(LispObject lits, int nargs, ...)
{   int sh;
    ARG2("rightshift", x, y);
    if (!isFIXNUM(y)) return error1("Bad argument for rightshift", y);
    sh = (int)qfixnum(y);
    UNARYINTOP("rightshift", x);
}

\end{verbatim}}
