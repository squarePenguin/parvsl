\chapter{Some implementation tricks and extensions}
The main version of {\vsl} emphasises simplicity over capability.
This chapter discusses a couple of alternative implementation
strategies that could extend it. I will refer to an extended version
of {\vsl} including the ideas discussed here as {tx vslplus}.


\section{Conservative Garbage Collection}
{\vsl} has a garbage collector that requires
that all references into the heap can be unambiguously identified. This
constraint complicates the code, leads to the need for extensive use of
a special Lisp stack and will make Lisp compilation much messier than
if such a constraint did not apply. In 1988 Bartlett\cite{conservative}
and then Appell and Hanson\cite{conservative1} presented garbage
collection schemes that used a copying strategy but did not impose this
constraint. They both control the preservation of data that is subject to
(possible) reference from an ambiguous location in units of pages and by
so doing they reduce the overheap of tracking where such data is.
However at the same time they may tend to waste a little space and
(more seriously) set a limit on the largest size that an object can
have (and fit within one page).

So although I will refer back to those important early papers I will want
to arrange that I can cope with almost arbitrarily large vectors in my
heap. However I will not feel the need to allow for pointers that
refer within an object (ie to other than an object head) and I will view
values that have the {\vsl} tag-bits removed as valid. So in my first
implementation at least I will accept that if the user has code that
writes something like
{\small\begin{verbatim}
    LispObject a = ... // eg a string
    ... qstring(a) ...
\end{verbatim}}
and a compiler makes optimisations that compute and later re-use the
untagged address that the {\tx qstring} macro generates then there could
be pain.

Well actually it will be easy for me to take any ambiguous
value and discard its tag bits before checking if it might be a
pointer into the heap, and for the benefit of {\tx qstring} and
{\tx elt} view such a value as pinning the object whose address was
just before it -- or possibly scan back from it to the head of the
previous object in memory and then untagged references even into the
middle of arrays would become safe. That could make the code
safer in the context of unduly optimising compilers and would
probably not lead to verty much additional memory waste, so it
becomes the new policy. Whenever I talk later down about checking if
a reference is to the start of an object I have to mean scan back
to the start of the previous object and treat the reference as pinning]
that.

\subsection{What are the ambiguous bases?}
The key problem that a Conservative Garbage Collector addresses is
support for a system where some pointers into the heap are stored in
locations where one can not be certain that they contain valid data. This
could include cases where pointer-containing memory is interleaved
with memory containing arbitrary other material and where it is not feasible
to maintain maps or table showing exactly what is where. For a mark-and-sweep
garbage collection strategy all that would be needed to cope with this would
be to have the mark process in the garbage collector mark from every
location that could possibly contain a pointer. The code would have to be
robust against data that looked superficially like a pointer but was to an
invalid address. That includes addresses outside the computer's memory,
outside the heap, inside the heap but into the middle of an object (rather
than pointing at its head) and misaligned pointers. But in that case
the sweep phase of the collector would not need any adjustment at all.

For a copying collector things are harder, because if data is copied
from an old half-space to a new one then any pointer to it must be changed
to track it. ``Ambiguous bases'' are memory locations that might contain
a pointer in the heap, and the value stored in one is an ``ambiguous
reference''. Any such reference could be a pointer but might be
merely an integer with an unfortunate value, or part of a string or
other binary data. So for safety its value must not be changed by the
garbage collector, but the heap item it (potentially) refers to must be
preserved. Thus items in the heap referred to directly from an
ambiguous base must be handled as for a mark-and-sweep collector.

Any item in the heap that is only referenced from locations that are
not ambiguous can, however, be copied or moved.

The only ambiguous bases I need to consider in {\tx vsl} are on the
C stack. All other referencess can be taken as fully under control, and the
garbage collector must and will (and can) have information showing all
the static variables that can ever hold references. Indeed the {\tx vsl}
code ensures this by keeping its (static) roots in an array called
{\tx bases} making it truly easy identify and process them.

I need to have code that
can be assured of finding everything that is (or that should be) on the
stack. This may involve some trickery in the fact of variables that are
stored in registers not (immediately) on the stack\ldots but I hope that
arranging that the garbage collector itself declared and uses a large number
of variables will ensure that if flushes all earlier callee-save values
to the real stack.

One thing that emerges from this is that the original {\tx vsl} scheme
for floating point numbers (which allocates floats downwards from the top
of the heap and everything else upwards from the bottom) becomes untenable.
To see this imagine a case where a tight loop allocated very many floats and
nothing else. Almost the whole heap is full of floats when garbage collection
is triggered. Now suppose an ambiguous pointer pins one of the last to be
allocated (and this in fact seems a very plausible situation). Then an
address low within the heap has to end up storing a float. If the
segregation between floats and other things is to be retained then this
will really limit the space available for anything else. So an early
change to {\vsl} needed to support conservative garbage collection has to be
to store floats in a way where they can mingle freely with other items in
the heap. There are two potential ways of doing this -- either every float
is stored with a header word (doubling the space it consumes and
slighly slowing allocation) or a bitmap is needed that flags which words
in the entire heap contain floats (probably with similar speed impact).

Let me compare the space demands of each. If there $f\%$ of the heap is
consumed by floats in the current scheme then adding headers consumes an
additional $f\%$. For the bitmap approach I need one bit for every 64-bits of
heap, leading to a $1.5\%$ storage overhead regardless of whether floats are
being used or not. Almost all of the time in most Lisp code there will hardly
be any floats at all, but when they are in use the percentage heap occupation
that may matter most relates to floats that have just been allocated, used
and discarded. The bitmap approach could reduce the frequency of garbage
collections triggered during heavy floating point use by up to a factor of
two. Heavy use of declarations in the code and a ``sufficiently clever''
compiler that managed to do a lot of floating point arithmetic un-boxed
could also make that saving. But I end up judging that the memory-use for
the bitmap is in general a low overhead during general use as it
provides a useful saving during stretches of execution that allocate
many million floats.

\subsection{Overall strategy}
\begin{enumerate}
\item For each ambiguous base, look at content. If it does not refer to
heap data that I had previously allocated ignore it. Otherwise
tag that data as immovable.

To implement this I first do a range check on the address to see if its
lies within the area of the current active heap-half. I need to check
tag-bits and alignment for sanity. I will maintain a bit-map that has
a bit set to mark the first address of every object in the heap. Because
all objects are doubleword aligned this needs one bit for every 8-bytes
of heap, so this is only a 1 in 64 (1.5\%) burden on memory - and the cost
of setting that bit during a {\tx cons} (and other allocation operations) is
not terribly severe. That gives a really simple way that I can end up
certain that I know when an ambiguous pointer truly points at the head
of an object. If the object was a symbols it started with a flag word,
and if it was a vector-like atom it would have a header word -- and in each
of those cases it would be easy to find a spare bit to use to mark it
as immovable (I will often use the term ``pinned'' here to describe that
state). However cons cells and headerless floats do not have spare bits,
so the bitmap scheme here is the simplest option.

I want to mention a range of possibilities here even if I end up discarding
them. It seems probably that {\em usually} there will only be a modest
number of items that end up pinned. In that case using a hash table to
record where they are could seem attractive. The reason I do not go down
that route is twofold: I do not have a reliable bound on how large that
table would be, and bitmap tagging is already a guaranteed unit-cost solution.
However a hash table could be attractive in that if multiple ambigiuous
bases contain the same value it could help some processing deal with each
just once. I think at present I do not believe that to be a big enough
benefit to be worth fussing with.


\item  Note that I must not start copying {\em anything} until I have
identified
everything that must be pinned. Items that are pinned
must never move! So run the normal copying GC except that when a
pinned is reached it is just left in place. The items within
the pinned item are not copied at this stage because to do so would
involve stack use within the garbage collector, and that is undesirable.

Then
re-scan the ambiguous bases and copy items that are referred to from
pinned items. Note that this is not moving the
pinned items themselved, just their contents. Also there is
a requirement here that the set of and values in ambiguous bases must
not have changed during garbage collection so far.

Since the contents of an item must be copied exactly once
and since several ambiguous bases may refer to the same location it will
be necessary to be able to tell if something has been processed. This is
not very different from the challenge already facing the existing copying
garbage collector. Specifically floating point values will (still) use
a bitmap.

\item After copying there are some pinned items in the old heap. These
must remain identifiable. Since it will be necessary to do a linear scan
of the bitmap of pinned items at some stage the neatest idea will be
to retain that bitmap and make allocation inspect it. When needing to
skip past pinned items this will involve looking at their headers. Because
in at least very many cases of garbage collection the stack will only
be a few thousand words deep but the heap might be millions of words
long it can be expected that pinned items are sparse in the heap, and
in particular that there will be long gaps between adjacent pinned
words. If the bitmap is often inspected word by word these runs can be
skipped over in blocks of 32, speeding up the sweep usefully.


\end{enumerate}


\subsection{Overview of bitmap states at start and end of collection}
\begin{description}
\item[Active space at start]
~

\begin{enumerate}
\item A map that has a bit set showing the location of every
headerless floating point value in the heap.
\item A map that will be used to mark floats. Clear it at start of GC.
\item A map that indicates that start address of each item in the heap.
\item The map that had marked pinned addresses from the previous
collection that were hence locations that could not be allocated
during the recent phase of computation. Since the garbage collection is
now starting (possibly because the heap was full) the data in this bitmap
is no longer required, so the map is cleared and can be used to tag
locations pinned at this time.
\end{enumerate}
\item[Second space at start]
~

\begin{enumerate}
\item A map that has a bit set showing the location of every
headerless floating point value in the heap. The only data that
can be in this space will be material pinned during the previous
collection.
\item a map that would be used to mark floats. Not used! Thus the
space for maps can be reduced by having only a half-sized one for
marking floats so it doe snot cover the second half-space.
\item A map available to indicate that start address of each item in the heap.
Again the only items in the heap are ones that were pinned last time, so this
can be a copy of the pinning bitmap left over from the previous collection.
\item The map that had marked pinned addresses from the previous
collection. The data from these is still in place and must be avoided while
cupying into the second space, but that purpose can be achieved via the
(copy of) the data as above. Copying can proceed and set a new start-of-object
bit while avoding allocating where ons is already set. So this map can be
cleared at the start of collection. Well more precisely the object-start
and pinned maps can be exchanged to avoid the need for any bulk copying then
the map to be used as the new pinning map can be cleared.
\end{enumerate}
\item[Old active space at end]
~

\begin{enumerate}
\item A map that has a bit set showing the location of every
headerless floating point value in the heap. To be ready for next
time I think this needs masking with the map of pinned items in effect
to discard all non-pinned floats on the old space.
\item floating point mark bits. Not needed at end of collection.
\item A map that indicates that start address of each item in the heap.
The data in this will no longer be needed
\item Pinned data from this collection. Required for the future since
the addresses so marked must be avoided by allocation within this
half-space when that done during the next garbage collection.
\end{enumerate}
\item[Second space at end]
~

\begin{enumerate}
\item Floating point map, now showing the location of both long-term
pinned floats and ones that have just been copied to this space.
\item (floating point mark bits)
\item Object start adddresses.
\item Pinned data. The information here is no longer required. 
\end{enumerate}
\end{description}

\subsection{Interaction with saving heap images}
There can be two situations where an image of the state of Lisp must be
saved to a file. The first is when it represents a checkpoint and
execution must continue after the image has been written. In this case
all pinned items must be written, and the image on disc thus ends up
(potentially) sparse. The pinned items have to be written out not because
the ambiguous pointers that led to their pinning will still be present
in a frest run of {\tx vslplus} but because ordinary unambigiuous references
to them are also liable to exist.

The second case is when the system will stop immediatly after creating
an image file, and in this case it can be arranged that a fully compacted
image file results.
I can do a garbage collection that avoids marking from
ambiguous roots during {\tx preserve} since
at that stage there ought not to be any important stack to mark. As a result
no items will remain pinned at the end of garbage collection.
Well in such cases I need just
a bit more -- I should ensure that the new space that I copy to does not
contain any items that had been pinned during the previous GC.
This can be achieved by performing two garbage
collections so I end up with a fully compacted heap to save.

There are further issues that arise if there is a desire to expand the heap
at run-time. If this is to be done other than when the stack is empty
it is important that all existing parts of heap memory remain
in position so that pinned items remain accessible. So if run-time
heap expansion is to be considered it becomes necessary to structure
both halves of the heap into segments with initial
segments allocated at system start-up and new ones potentially added
dynamically. Of course the format for saved heap images then has to
allow for that! The size selected for segments will limit the size of the
largest vector that can be allocated, and space will sometimes end up
wasted near the end of a segment when a large item will only fit in the
next one.

\subsection{Wrapping up}
With the rework described here many of the existing uses of {\tx push} and
{\tx pop} in the original code become unnecessary, since Lisp references kept
in simple C local variables become garbage-collection-safe. The speed
of garbage collection is slightly impacted, and around $5\%$ extra memory
is needed for bitmaps, but experience has shown that conservative garbage
collection schemes manage to recycle memory very nearly as effectively as
simple precise ones, and they simplify not just the rest of the inner workings
of the system they are built to work with, but the ability of it to
link to other bodies of code that they may wish to exchange data with. It is
overall a good policy to adopt. 


\section{Managing compiled code}
When a Lisp system compiles bits of Lisp into native code for the machine
that is in use it finds itself first treating the generated code as data (while
it fills in all the bits) and then as executable code. Modern operating
systems tend to try to protect against such behaviour since the dynmamic
creation of code is a technique used in various sorts of malware. So
typically special operating system calls are called for to mark some
region of memory as available for use in this way.

% @@@@
