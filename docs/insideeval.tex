\chapter{Evaluation}


% @@@@@ incomplete

% At present I just put in the C code that includes stuff Ii will need to
% describe here.

{\small\begin{verbatim}
#define qflags(x) (((LispObject *)((x)-tagSYMBOL))[0])
#define qvalue(x) (((LispObject *)((x)-tagSYMBOL))[1])
#define qplist(x) (((LispObject *)((x)-tagSYMBOL))[2])
#define qpname(x) (((LispObject *)((x)-tagSYMBOL))[3])
#define qdefn(x)  (((void **)     ((x)-tagSYMBOL))[4])
#define qlits(x)  (((LispObject *)((x)-tagSYMBOL))[5])

// Bits within the flags field of a symbol. Uses explained later on.

#define flagTRACED    0x080
#define flagSPECFORM  0x100
#define flagMACRO     0x200

int unwindflag = 0;

#define unwindNONE      0
#define unwindERROR     1
#define unwindBACKTRACE 2
#define unwindGO        3
#define unwindRETURN    4

int backtraceflag = -1;
#define backtraceHEADER 1
#define backtraceTRACE  2

LispObject error1(const char *msg, LispObject data)
{   if ((backtraceflag & backtraceHEADER) != 0)
    {   linepos = printf("\n+++ Error: %s: ", msg);
        errprint(data);
    }
    unwindflag = (backtraceflag & backtraceTRACE) != 0 ? unwindBACKTRACE :
                 unwindERROR;
    return nil;
}

typedef LispObject specialform(LispObject data, LispObject x);
typedef LispObject lispfn(LispObject data, int nargs, ...);

LispObject applytostack(int n)
{
// Apply a function to n arguments.
// Here the stack has first the function, and then n arguments. The code is
// grim and basically repetitive, and to avoid it being even worse I will
// expect that almost all Lisp functions have at most 4 arguments, so
// if there are more than that I will pass the fifth and beyond all in a list.
    LispObject f, w;
    int traced = (qflags(sp[-n-1]) & flagTRACED) != 0;
    if (traced)
    {   int i;
        linepos = printf("Calling: ");
        errprint(sp[-n-1]);
        for (i=1; i<=n; i++)
        {   linepos = printf("Arg%d: ", i);
            errprint(sp[i-n-1]);
        }
    }
    if (n >= 5)
    {   push(nil);
        n++;
        while (n > 5)
        {   pop(w);
            TOS = cons(TOS, w);
            n--;
        }
    }
    switch (n)
    {   case 0: f = TOS;
            w = (*(lispfn *)qdefn(f))(qlits(f), 0);
            break;
        case 1:
        {   LispObject a1;
            pop(a1);
            f = TOS;
            w = (*(lispfn *)qdefn(f))(qlits(f), 1, a1);
            break;
        }
        case 2:
        {   LispObject a1, a2;
            pop(a2)
            pop(a1);
            f = TOS;
            w = (*(lispfn *)qdefn(f))(qlits(f), 2, a1, a2);
            break;
        }
        case 3:
        {   LispObject a1, a2, a3;
            pop(a3);
            pop(a2)
            pop(a1);
            f = TOS;
            w = (*(lispfn *)qdefn(f))(qlits(f), 3, a1, a2, a3);
            break;
        }
        case 4:
        {   LispObject a1, a2, a3, a4;
            pop(a4);
            pop(a3);
            pop(a2)
            pop(a1);
            f = TOS;
            w = (*(lispfn *)qdefn(f))(qlits(f), 4,
                                      a1, a2, a3, a4);
            break;
        }
        case 5:
        {   LispObject a1, a2, a3, a4, a5andup;
            pop(a5andup);
            pop(a4);
            pop(a3);
            pop(a2)
            pop(a1);
            f = TOS;
            w = (*(lispfn *)qdefn(f))(qlits(f), 5,
                                      a1, a2, a3, a4, a5andup);
            break;
        }
        default:disaster(__LINE__);
            return nil;
    }
    pop(f);
    if (unwindflag == unwindBACKTRACE)
    {   linepos = printf("Calling: ");
        errprint(f);
    }
    else if (traced)
    {   push(w);
        prin(f);
        linepos += printf(" = ");
        errprint(w);
        pop(w);
    }
    return w;
}

LispObject interpret(LispObject def, int nargs, ...);
LispObject Lgensym(LispObject lits, int nargs, ...);

LispObject eval(LispObject x)
{   while (isCONS(x) && isSYMBOL(qcar(x)) && (qflags(qcar(x)) & flagMACRO))
    {   push2(qcar(x), x);
        x = applytostack(1);  // Macroexpand before anything else.
        if (unwindflag != unwindNONE) return nil;
    }
    if (isSYMBOL(x))
    {   LispObject v = qvalue(x);
        if (v == undefined) return error1("undefined variable", x);
        else return v;
    }
    else if (!isCONS(x)) return x;
// Now I have something of the form
//     (f arg1 ... argn)
// to process.
    {   LispObject f = qcar(x);
        if (isSYMBOL(f))
        {   LispObject flags = qflags(f), aa, av;
            int i, n = 0;
            if (flags & flagSPECFORM)
            {   specialform *fn = (specialform *)qdefn(f);
                return (*fn)(qlits(f), qcdr(x));
            }
// ... else not a special form...
            if (qdefn(f) == NULL) return error1("undefined function", f);
            aa = qcdr(x);
            while (isCONS(aa))
            {   n++;             // Count number of args supplied.
                aa = qcdr(aa);
            }
            aa = qcdr(x);
            push(f);
// Here I will evaluate all the arguments for the function, leaving the
// evaluated results on the stack.
            for (i=0; i<n; i++)
            {   push(aa);
                av = eval(qcar(aa));
                if (unwindflag != unwindNONE)
                {   while (i != 0)  // Restore the stack if unwinding.
                    {   pop(aa);
                        i--;
                    }
                    pop2(aa, aa);
                    return nil;
                }
                aa = qcdr(TOS);
                TOS = av;
            }
            return applytostack(n);
        }
        else if (isCONS(f) && qcar(f) == lambda)
        {   LispObject w;
            push(x);
            w = Lgensym(nil, 0);
            pop(x);
            qdefn(w) = (void *)interpret;
            qlits(w) = qcdr(qcar(x));
            return eval(cons(w, qcdr(x)));
        }
        else return error1("invalid function", f);
    }
}

LispObject interpretspecform(LispObject lits, LispObject x)
{   // lits should be ((var) body...)
    LispObject v;
    if (!isCONS(lits)) return nil;
    v = qcar(lits);
    lits = qcdr(lits);
    if (!isCONS(v) || !isSYMBOL(v = qcar(v))) return nil;
    push2(qvalue(v), v);
    qvalue(v) = x;
    lits = Lprogn(nil, lits);
    pop2(v, qvalue(v));
    return lits;
}

// Special forms are things that do not have their arguments pre-evaluated.

LispObject Lquote(LispObject lits, LispObject x)
{   if (isCONS(x)) return qcar(x);
    else return nil;
}

LispObject Lcond(LispObject lits, LispObject x)
{
//   Arg is in form
//      ((predicate1 val1a val1b ...)
//       (predicate2 val2a val2b ...)
//       ...)
    while (isCONS(x))
    {   push(x);
        x = qcar(x);
        if (isCONS(x))
        {   LispObject p = eval(qcar(x));
            if (unwindflag != unwindNONE)
            {   pop(x);
                return nil;
            }
            else if (p != nil)
            {   pop(x);
                return Lprogn(nil, qcdr(qcar(x)));
            }
        }
        pop(x);
        x = qcdr(x);
    }
    return nil;
}


#define ARG0(name)              \
    if (nargs != 0) return error1s("wrong number of arguments for", name)

#define ARG1(name, x)           \
    va_list a;                  \
    LispObject x;               \
    if (nargs != 1) return error1s("wrong number of arguments for", name); \
    va_start(a, nargs);         \
    x = va_arg(a, LispObject);  \
    va_end(a)

#define ARG2(name, x, y)        \
    va_list a;                  \
    LispObject x, y;            \
    if (nargs != 2) return error1s("wrong number of arguments for", name); \
    va_start(a, nargs);         \
    x = va_arg(a, LispObject);  \
    y = va_arg(a, LispObject);  \
    va_end(a)

#define ARG3(name, x, y, z)     \
    va_list a;                  \
    LispObject x, y, z;         \
    if (nargs != 3) return error1s("wrong number of arguments for", name); \
    va_start(a, nargs);         \
    x = va_arg(a, LispObject);  \
    y = va_arg(a, LispObject);  \
    z = va_arg(a, LispObject);  \
    va_end(a)

#define ARG0123(name, x, y, z)                        \
    va_list a;                                        \
    LispObject x=NULLATOM, y=NULLATOM, z=NULLATOM;    \
    if (nargs > 3) return error1s("wrong number of arguments for", name); \
    va_start(a, nargs);                               \
    if (nargs > 0) x = va_arg(a, LispObject);         \
    if (nargs > 1) y = va_arg(a, LispObject);         \
    if (nargs > 2) z = va_arg(a, LispObject);         \
    va_end(a)

LispObject Lcar(LispObject lits, int nargs, ...)
{   ARG1("car", x);  // Note that this WILL take car of a bignum!
    if (isCONS(x)) return qcar(x);
    else return error1("car of an atom", x);
}

\end{verbatim}}
