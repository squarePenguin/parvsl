\chapter{Versions of Lisp}
The two most widely referenced Lisp-family dialects today are
Common Lisp\cite{ANSILisp} and Scheme\cite{Scheme}. There are
proper standards documents that define each of these. Common Lisp is
huge and attempts to encompass absolutely everything, while Scheme is
sometimes described as ``minimalist''. 

Neither of these dialects is used here. That situation clearly needs
justification. The short form of this is that Common Lisp has so many
facilities that a subset short enough to explain will omit so much of
what would make it ``Common'' as to be ridiculous.
Scheme on the other hand is a fairly compact language, but it insists
on ``correct'' treatment of various key features (notably variable scopes,
tail recursion and continuations) that tend to make an implementation
larger, slower and harder to explain.

There have been many other Lisp dialects and implementations. Common Lisp
can be seen as the successor to Maclisp and its variants embodied in various
special-purpose Lisp computers built at MIT. HLISP\cite{hlisp} was developed
\begin{wrapfigure}{r}{3.5in}
{\centering
\includegraphics[width=3.4in]{goto-and-flats.eps}}
\caption{Eiichi Goto and the FLATS machine}
\end{wrapfigure} by Eiichi Goto in Japan with a key feature that memory was saved by using
hashing to implement the {\tx cons} operation so that as much data as
possible could be shared. HLISP led to the development of the FLATS\cite{flats}
computer which combined high performance for both Lisp-style symbolic
computation with good numeric capability -- with hardware acceleration for
hashing and associative operations that made HLISP special.

At the other extreme of the size and performance scale, muLisp (in
due course distributed by the Soft Warehouse Inc) started as a 2 Kbyte
implementation manually toggled into the memory of a 4 Kbyte computer
based on an 8080 processor\cite{Imsai}.



One of the notable implementations of Common Lisp is {\tx gcl} (GNU Common
Lisp). Its full source can be fetched via (for instance) the Free
Software Foundation's website at
{\tx http://www.gnu.org/software/gcl}. The source archive
contains around 180,000 lines of Lisp code, as well as significant amounts
of C code that among other things provide support for arbitrary precision
arithmetic. The ``Visible'' Lisp used here (\vsl) has as its main
source just over 3000 lines of C code, which is already quite enough to
try to explain.

Scheme is smaller, to the extent that some would claim that the basic
version was only suitable for teaching not for real use. There are
an enormous number of implementations of if: one list names 78. But the
vast majority of these are also much bulkier than \vsl{} and hence would be
harder to explain.

The alternative approach used here has been to go a little further back in
history and use the Standard Lisp dialect that originated at the University
of Utah. There were two stages in its definition. 
The report defining an initial version of it was by Tony Hearn in
1969\cite{StdLisp1} and was his attempt to codify a Lisp dialect that
could provide portability for his Reduce algebra system\cite{Reduce}. A
decade later Marti, Hearn and Griss produced a substantial update\cite{StdLisp2}.
Griss, Benson and Maguire released PSL\cite{PSL} as a Portable implementation
of Standard Lisp a couple of years later, and others involved with the
Reduce project followed on with at least Cambridge Lisp\cite{Camlisp},
then CSL\cite{CSL} and a Java-coded version known as Jlisp\cite{Jlisp}. These
days PSL, CSL and Jlisp can all be found on the website
{\tx http://reduce-algebra.sourceforge.net} and all are both in use and
continued development.

By following Standard Lisp the code and explanations here steer a line between
excess complexity and being so cut down and simple that they have no
real relevance. The simplicity of \vsl can be assessed by just looking at
the fairly small amount of C code that implemented it, and this code
implements the most important facilities that any Lisp variant will require.
The power of \vsl{} can be seen from the fact that it can be used to build
the roughly 350,000 lines of code that make up the whole of the Reduce algebra
system, with almost every part of that code working as expected. There are
three key limitations that mean that Reduce built in top of \vsl{} may not
be suitable for heavy use. The first is that \vsl{} only supports
integers with values up to 9223372036854775807 in a tolerably
efficient manner, while Reduce needs fully
arbitrary precision arithmetic in many places. The second is that \vsl{}
provides just an interpreter for Lisp, not a compiler, and as a result
calculations performed using it will be somewhat slow. The third issue is that
\vsl{} only supports very basic interaction with the user via the keyboard, and
many of today's users would expect a much richer user-interface. But all those
limitations admitted, it remains that \vsl{} can support Lisp applications
that run up in to the hundreds of thousands of lines of source code!

Despite the adoption of the Standard Lisp dialect for \vsl{} it has to be
conceded that Common Lisp can not properly be ignored. It remains the fact that
a full implementation would be much too large to explain, and indeed even
full coverage of the language from the perspective of a user would need
a seriously lengthy and probably rather dull book. So three things are
done here to provide support for a move onward from \vsl{} into use of
Common Lisp. First all the example programs are provided in Common Lisp
for as part of the down-loadable resources to accompany this book. The
code there has comments in it noting such special features of Common Lisp
as appear. Secondly a modified version of \vsl{} that provides compatibility
with the core of Common Lisp is also provided for download -- it is known
as \vcl. \vcl{} is naturally far from complete as a Common Lisp, and its
documentation here is much less complete than that for \vsl{}, but it can form
a bridge to use of Big Lisp. Many of the features of \vsl{} are implemented
as Lisp code that builds on a fairly small number of primitive operations.
The final aspect of Common Lisp support is that the most interesting or
important of these are described and explained as an extended set of
fragments of Lisp sample code. This explanation can give both insight into
some of the power of Common Lisp and some of the pain it gives any
person trying to provide a high performance implementation.

A definite stance taken with \vsl{} is that it is not intended to be
viewed as a fully-finished Lisp system merely for use. It is more a starting
point for project work that will extend it. Those who are convinced that
Common Lisp or Scheme would be better can use much of the core technology
within \vsl{} and adapt it to implement the dialect that they are keen on.
It would be essentially trivial to just change the names that \vsl{} uses
for the Lisp functions that it supplies. For instance Scheme uses the names
{\tx set!} and {\tx equal?} where \vsl{} uses {\tx setq} and {\tx equal},
and Common Lisp used {\tx defun} where \vsl{} used {\tx de}. Adding extra
functions, and even data-types should be straightforward. Deeper semantic
differences between the dialects would naturally lead to more challenging
re-work.

Anybody who became enthusiastic about using \vsl{} for algebraic computation
might retrofit full-speed arbitrary prevision arithmetic. If they wanted to match the
arithmetic capability of Common Lisp they would also need to add complex
numbers and a full range of elementary functions working on them.

Lisp is a very natural language for compiler-building, and a small and
fairly tidy version like \vsl{} could provide a good starting point for
a project to build one. One of the examples given later here may provide
some sort of a start, but Appendix 3 in Berkeley and Bobrow\cite{BandB} (one
of many important historical Lisp books now available on-line) includes
a listing of a complete real Lisp compiler -- albeit for a machine now
long defunct. There is no reason to believe that a compiler for \vsl{} would
not transform it into a system speed-competitive with all other current
Lisps. The compiler in Berkeley and Bobrow is maybe 1500 lines of
code, and the key parts of if that do the central interesting bits of the work are maybe
only around 500 lines. This is not an outrageously huge project.

User-interface refinement is something that could generate an almost
unlimited amount of activity. Before starting on a project in this area it
would make sense to review what others have done. One possible
place to start would be to investigate LispWorks\cite{LispWorks},
which provides an integrated cross-platform development environment.




