
\chapter{Checkpoint and restore}



% @@@@@ incomplete

{\small\begin{verbatim}

// A saved image will start with a word that contains the following 32-bit
// code. This can identify the byte-ordering and word-width of the system
// involved, so protects against attempts to reload an image on a machine
// other than the one it was build on. If I was being more proper I would
// include a version number as well.

#define FILEID (('v' << 0) | ('s' << 8) | ('l' << 16) |   \
                (('0' + sizeof(LispObject)) << 24))

static const char *imagename = "vsl.img";

LispObject Lpreserve(LispObject lits, int nargs, ...)
{
// preserve can take either 0 or 1 args. If it has a (non-nil) arg that will
// be a startup function for the image when restored.
    FILE *f;
    ARG0123("preserve", x,y,z);
    if (y != NULLATOM || z != NULLATOM)
        return error1s("wrong number of arguments for", "preserve");
    restartfn = (x == NULLATOM ? nil : x);
    f = fopen(imagename, "wb");
    if (f == NULL) return error1("unable to open image for writing", nil);
    headerword = FILEID;
    reclaim(); // To compact memory.
// I write this stuff out as a bunch of bytes, since I only intend to
// re-read it on exactly the same computer.
    saveinterp = (LispObject)(void *)interpret;
    saveinterpspec = (LispObject)(void *)interpretspecform;
    fwrite(nonbases, 1, sizeof(nonbases), f);
    fwrite(bases, 1, sizeof(bases), f);
    fwrite(obhash, 1, sizeof(obhash), f);
    fwrite((void *)heap1base, 1, (size_t)(fringe1-heap1base), f);
    fwrite((void *)fpfringe1, 1, (size_t)(heap1top-fpfringe1), f);
    fclose(f);
// A cautious person would have checked for error codes returned by the
// above calls to fwrite and close. I omit that here to be concise.
    return nil;
}

jmp_buf restart_buffer;
int coldstart = 0;

LispObject Lrestart_csl(LispObject lits, int nargs, ...)
{   LispObject save = lispout;
    ARG0123("restart-csl", x, y, z);
    if (z != NULLATOM)
        return error1s("wrong number of arguments for", "restart-csl");
    coldstart = 0;
    if (x == nil || x == NULLATOM) coldstart = 1, x = nil;
    if (y == NULLATOM) x = cons(x, nil);
    else x = list2star(x, y, nil);
    boffop = 0;
    lispout = -2;
    prin(x);
    lispout = save;
    longjmp(restart_buffer, 1);
}


struct defined_functions
{   const char *name;
    int flags;
    void *entrypoint;
};

struct defined_functions fnsetup[] =
{
// First the special forms
    {"quote",      flagSPECFORM, (void *)Lquote},
    {"cond",       flagSPECFORM, (void *)Lcond},
    {"and",        flagSPECFORM, (void *)Land},
    {"or",         flagSPECFORM, (void *)Lor},
    {"de",         flagSPECFORM, (void *)Lde},
    {"df",         flagSPECFORM, (void *)Ldf},
    {"dm",         flagSPECFORM, (void *)Ldm},
    {"setq",       flagSPECFORM, (void *)Lsetq},
    {"progn",      flagSPECFORM, (void *)Lprogn},
    {"prog",       flagSPECFORM, (void *)Lprog},
    {"go",         flagSPECFORM, (void *)Lgo},
// The following are implemented as special forms here because they
// take variable or arbitrary numbers of arguments - however they all
// evaluate all their arguments in a rather simple way, so they
// could be treated a sorts of "ordinary" function.
    {"list",       flagSPECFORM, (void *)Llist},
    {"list*",      flagSPECFORM, (void *)Lliststar},
    {"iplus",      flagSPECFORM, (void *)Lplus},
    {"itimes",     flagSPECFORM, (void *)Ltimes},
    {"ilogand",    flagSPECFORM, (void *)Llogand},
    {"ilogor",     flagSPECFORM, (void *)Llogor},
    {"ilogxor",    flagSPECFORM, (void *)Llogxor},
// Now ordinary functions. I have put these in alphabetic order.
    {"iadd1",      0,            (void *)Ladd1},
    {"apply",      0,            (void *)Lapply},
    {"atan",       0,            (void *)Latan},
    {"atom",       0,            (void *)Latom},
    {"bignump",    0,            (void *)Lbignump},
    {"boundp",     0,            (void *)Lboundp},
    {"car",        0,            (void *)Lcar},
    {"cdr",        0,            (void *)Lcdr},
    {"char-code",  0,            (void *)Lcharcode},
    {"iceiling",   0,            (void *)Lceiling},
    {"close",      0,            (void *)Lclose},
    {"code-char",  0,            (void *)Lcodechar},
    {"compress",   0,            (void *)Lcompress},
    {"cons",       0,            (void *)Lcons},
    {"cos",        0,            (void *)Lcos},
    {"idifference",0,            (void *)Ldifference},
    {"idivide",    0,            (void *)Ldivide},
    {"eq",         0,            (void *)Leq},
    {"equal",      0,            (void *)Lequal},
    {"error",      0,            (void *)Lerror},
    {"errorset",   0,            (void *)Lerrorset},
    {"eval",       0,            (void *)Leval},
    {"exp",        0,            (void *)Lexp},
    {"explode",    0,            (void *)Lexplode},
    {"explodec",   0,            (void *)Lexplodec},
    {"ifix",       0,            (void *)Lfix},
    {"ifixp",      0,            (void *)Lfixp},
    {"ifloat",     0,            (void *)Lfloat},
    {"floatp",     0,            (void *)Lfloatp},
    {"ifloor",     0,            (void *)Lfloor},
    {"gensym",     0,            (void *)Lgensym},
    {"igeq",       0,            (void *)Lgeq},
    {"get",        0,            (void *)Lget},
    {"getd",       0,            (void *)Lgetd},
    {"gethash",    0,            (void *)Lgethash},
    {"getv",       0,            (void *)Lgetv},
    {"igreaterp",  0,            (void *)Lgreaterp},
    {"ileftshift", 0,            (void *)Lleftshift},
    {"ileq",       0,            (void *)Lleq},
    {"ilessp",     0,            (void *)Llessp},
    {"load-module",0,            (void *)Lrdf},
    {"log",        0,            (void *)Llog},
    {"ilognot",    0,            (void *)Llognot},
    {"iminus",     0,            (void *)Lminus},
    {"iminusp",    0,            (void *)Lminusp},
    {"mkhash",     0,            (void *)Lmkhash},
    {"mkvect",     0,            (void *)Lmkvect},
    {"null",       0,            (void *)Lnull},
    {"inumberp",   0,            (void *)Lnumberp},
    {"oblist",     0,            (void *)Loblist},
    {"onep",       0,            (void *)Lonep},
    {"open",       0,            (void *)Lopen},
    {"plist",      0,            (void *)Lplist},
    {"preserve",   0,            (void *)Lpreserve},
    {"prin",       0,            (void *)Lprin},
    {"princ",      0,            (void *)Lprinc},
    {"print",      0,            (void *)Lprint},
    {"printc",     0,            (void *)Lprintc},
    {"put",        0,            (void *)Lput},
    {"puthash",    0,            (void *)Lputhash},
    {"putv",       0,            (void *)Lputv},
    {"iquotient",  0,            (void *)Lquotient},
    {"rdf",        0,            (void *)Lrdf},
    {"rds",        0,            (void *)Lrds},
    {"read",       0,            (void *)Lread},
    {"readch",     0,            (void *)Lreadch},
    {"readline",   0,            (void *)Lreadline},
    {"iremainder", 0,            (void *)Lremainder},
    {"remhash",    0,            (void *)Lremhash},
    {"remprop",    0,            (void *)Lremprop},
    {"restart-csl",0,            (void *)Lrestart_csl},
    {"return",     0,            (void *)Lreturn},
    {"irightshift",0,            (void *)Lrightshift},
    {"rplaca",     0,            (void *)Lrplaca},
    {"rplacd",     0,            (void *)Lrplacd},
    {"set",        0,            (void *)Lset},
    {"sin",        0,            (void *)Lsin},
    {"sqrt",       0,            (void *)Lsqrt},
    {"stop",       0,            (void *)Lstop},
    {"stringp",    0,            (void *)Lstringp},
    {"isub1",      0,            (void *)Lsub1},
    {"symbolp",    0,            (void *)Lsymbolp},
    {"terpri",     0,            (void *)Lterpri},
    {"time",       0,            (void *)Ltime},
    {"trace",      0,            (void *)Ltrace},
    {"untrace",    0,            (void *)Luntrace},
    {"upbv",       0,            (void *)Lupbv},
    {"vectorp",    0,            (void *)Lvectorp},
    {"wrs",        0,            (void *)Lwrs},
    {"zerop",      0,            (void *)Lzerop},
    {NULL,         0,            NULL}
};

void setup()
{
// Ensure that initial symbols and functions are in place. Parts of this
// code are rather rambling and repetitive but this is at least a simple
// way to do things. I am going to assume that nothing can fail within this
// setup code, so I can omit all checks for error conditions.
    struct defined_functions *p;
    undefined = lookup("~indefinite-value~", 18, 1);
    qvalue(undefined) = undefined;
    nil = lookup("nil", 3, 1);
    qvalue(nil) = nil;
    lisptrue = lookup("t", 1, 1);
    qvalue(lisptrue) = lisptrue;
    qvalue(echo = lookup("*echo", 5, 1)) = interactive ? nil : lisptrue;
    qvalue(lispsystem = lookup("lispsystem*", 11, 1)) =
       list2star(lookup("vsl", 3, 1), lookup("csl", 3, 1),
                 cons(lookup("embedded", 8, 1), nil));
    quote = lookup("quote", 5, 1);
    backquote = lookup("`", 1, 1);
    comma = lookup(",", 1, 1);
    comma_at = lookup(",@", 2, 1);
    eofsym = lookup("$eof$", 5, 1);
    qvalue(eofsym) = eofsym;
    lambda = lookup("lambda", 6, 1);
    expr = lookup("expr", 4, 1);
    subr = lookup("subr", 4, 1);
    fexpr = lookup("fexpr", 5, 1);
    fsubr = lookup("fsubr", 5, 1);
    macro = lookup("macro", 5, 1);
    input = lookup("input", 5, 1);
    output = lookup("output", 6, 1);
    pipe = lookup("pipe", 4, 1);
    qvalue(dfprint = lookup("dfprint*", 6, 1)) = nil;
    bignum = lookup("~bignum", 7, 1);
    qlits(lookup("load-module", 11, 1)) = lisptrue;
    qvalue(raise = lookup("*raise", 6, 1)) = nil;
    qvalue(lower = lookup("*lower", 6, 1)) = lisptrue;
    cursym = nil;
    work1 = work2 = nil;
    p = fnsetup;
    while (p->name != NULL)
    {   LispObject w = lookup(p->name, strlen(p->name), 1);
        qflags(w) |= p->flags;
        qdefn(w) = p->entrypoint;
        p++;
    }
}

void cold_setup()
{
// version of setup to call when there is no initial heap image at all.
    int i;
// I make the object-hash-table lists end in a fixnum rather than nil
// because I want to create the hash table before even the symbol nil
// exists.
    for (i=0; i<OBHASH_SIZE; i++) obhash[i] = tagFIXNUM;
    for (i=0; i<BASES_SIZE; i++) bases[i] = NULLATOM;
    setup();
// The following fields could not be set up quite early enough in the
// cold start case, so I repair them now.
    restartfn = qplist(undefined) = qlits(undefined) =
        qplist(nil) = qlits(nil) = nil;
}

LispObject relocate(LispObject a, LispObject change)
{
// Used to update a LispObject when reloaded from a saved heap image.
    switch (a & TAGBITS)
    {   case tagATOM:
           if (a == NULLATOM) return a;
        case tagCONS:
        case tagSYMBOL:
        case tagFLOAT:
            return a + change;
        default:
//case tagFIXNUM:
//case tagFORWARD:
//case tagHDR:
            return a;
    }
}

void warm_setup()
{
// The idea here is that a file called "vsl.img" will already have been
// created by a previous use of vsl, and it should be re-loaded.
    FILE *f = fopen(imagename, "rb");
    int i;
    LispObject currentbase = heap1base, change, *s;
    if (f == NULL)
    {   printf("Error: unable to open image for reading\n");
        exit(1);
    }
    if (fread(nonbases, 1, sizeof(nonbases), f) != sizeof(nonbases) ||
        headerword != FILEID ||
        fread(bases, 1, sizeof(bases), f) != sizeof(bases) ||
        fread(obhash, 1, sizeof(obhash), f) != sizeof(obhash))
    {   printf("Error: Image file corrupted or incompatible\n");
        exit(1);
    }
    change = currentbase - heap1base;
// Now I relocate the key addresses to refer to the CURRENT rather than
// the saved address map.
    heap1base  += change;
    heap1top   += change;
    fringe1    += change;
    fpfringe1  += change;
    if (fread((void *)heap1base, 1, (size_t)(fringe1-heap1base), f) !=
              (size_t)(fringe1-heap1base) ||
        fread((void *)fpfringe1, 1, (size_t)(heap1top-fpfringe1), f) !=
              (size_t)(heap1top-fpfringe1))
    {   printf("Error: Unable to read image file\n");
        exit(1);
    }
    fclose(f);
    for (i=0; i<BASES_SIZE; i++)
        bases[i] = relocate(bases[i], change);
    for (i=0; i<OBHASH_SIZE; i++)
        obhash[i] = relocate(obhash[i], change);
// The main heap now needs to be scanned and addresses in it corrected.
    s = (LispObject *)heap1base;
    while ((LispObject)s != fringe1)
    {   LispObject h = *s, w;
        if (!isHDR(h)) // The item to be processed is a simple cons cell
        {   *s++ = relocate(h, change);
            *s = relocate(*s, change);
            s++;
        }
        else              // The item is one that uses a header
            switch (h & TYPEBITS)
            {   case typeSYM:
                    w = ((LispObject)s) + tagSYMBOL;
                    // qflags(w) does not need adjusting
                    qvalue(w) = relocate(qvalue(w), change);
                    qplist(w) = relocate(qplist(w), change);
                    qpname(w) = relocate(qpname(w), change);
                    if (qdefn(w) == (void *)saveinterp)
                        qdefn(w) = (void *)interpret;
                    else if (qdefn(w) == (void *)saveinterpspec)
                        qdefn(w) = (void *)interpretspecform;
                    qlits(w)  = relocate(qlits(w), change);
                    s += 6;
                    continue;
                case typeSTRING: case typeBIGNUM:
// These sorts of atom just contain binary data so do not need adjusting,
// but I have to allow for the length code being in bytes etc.
                    w = (LispObject)s;
                    w += veclength(h);
                    w = (w + sizeof(LispObject) + 7) & ~7;
                    s = (LispObject *)w;
                    continue;
                case typeVEC: case typeEQHASH: case typeEQHASHX:
                    s++;
                    w = veclength(h);
                    while (w > 0)
                    {   *s = relocate(*s, change);
                        s++;
                        w -= sizeof(LispObject);
                    }
                    s = (LispObject *)(((LispObject)s + 7) & ~7);
                    continue;
                default:
                    // The spare codes!
                    disaster(__LINE__);
            }
    }
    setup(); // resets all built-in functions properly.
}

\end{verbatim}}
