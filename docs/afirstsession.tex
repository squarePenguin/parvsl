\chapter{A first session with Lisp}
To use the ``\vsl'' version of Lisp discussed here you will first have to
fetch it and ensure that it works on your computer. This should be
possible on almost any system that presents itself as a full computer,
and Appendix A gives explicit explanations for how to do this for
some of the most plausible platforms. Specifically it covers the case for
Raspberry Pi, an ordinary pc running Windows and a Macintosh. At present \vsl{}
is not directly usable on tablet-format machines, mobile phones or the like.
That is not because it could not be ported to them, but more because
the simple inner workings that it seeks to present would end up wrapped
in rather a lot of messy platform-specific code. When using \vsl{}
you will generally be interacting using a command prompt and keyboard
rather than any more modern-seeming graphical interface.


{\small\begin{verbatim}
(de factorial (n)
   (cond
      ((zerop n) 1)
      (t (times n (factorial (sub1 n))))))

(dotimes (i 40)
   (prin i)
   (princ blank)
   (print (factorial i)))
\end{verbatim}}

{\small\begin{verbatim}
0 1
1 1
2 2
3 6
4 24
5 120
6 720
7 5040
8 40320
9 362880
10 3628800
11 39916800
12 479001600
13 6227020800
14 87178291200
15 1307674368000
16 20922789888000
17 355687428096000
18 6402373705728000
19 121645100408832000
20 2432902008176640000
21 51090942171709440000
22 1124000727777607680000
23 25852016738884976640000
24 620448401733239439360000
25 15511210043330985984000000
26 403291461126605635584000000
27 10888869450418352160768000000
28 304888344611713860501504000000
29 8841761993739701954543616000000
30 265252859812191058636308480000000
31 8222838654177922817725562880000000
32 263130836933693530167218012160000000
33 8683317618811886495518194401280000000
34 295232799039604140847618609643520000000
35 10333147966386144929666651337523200000000
36 371993326789901217467999448150835200000000
37 13763753091226345046315979581580902400000000
38 523022617466601111760007224100074291200000000
39 20397882081197443358640281739902897356800000000
Value: nil
\end{verbatim}}


% @@@@@ incomplete


