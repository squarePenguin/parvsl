% @@@@ incomplete

% This may be too long and MAY need splitting into two chapters, one
% on printing and one on reading. On the other hand I may be able to
% remove junk and get it short enough to make sense!

{\begin{verbatim}


#define printPLAIN   1
#define printESCAPES 2

int linelength = 80, linepos = 0, printflags = printESCAPES;

#define MAX_LISPFILES 30
FILE *lispfiles[MAX_LISPFILES];
int32_t file_direction = 0, interactive = 0;
int lispin = 0, lispout = 1;

extern LispObject lookup(const char *s, int n, int createp);

void wrch(int c)
{
    if (lispout == -1)
    {   char w[4];
// This bit is for the benefit of explode and explodec
        LispObject r;
        w[0] = c; w[1] = 0;
        r = lookup(w, 1, 1);
        work1 = cons(r, work1);
    }
    else if (lispout == -2) boffo[boffop++] = c;
    else
    {   putc(c, lispfiles[lispout]);
        if (c == '\n')
        {   linepos = 0;
            fflush(lispfiles[lispout]);
        }
        else linepos++;
    }
}

int rdch()
{   LispObject w;
    if (lispin == -1)
    {   if (!isCONS(work1)) return EOF;
        w = qcar(work1);
        work1 = qcdr(work1);
        if (isSYMBOL(w)) w = qpname(w);
        if (!isSTRING(w)) return EOF;
        return *qstring(w);
    }
    else
    {   int c = getc(lispfiles[lispin]);
        if (c != EOF && qvalue(echo) != nil) wrch(c);
        return c;
    }
}

int gensymcounter = 1;

void checkspace(int n)
{   if (linepos + n >= linelength && lispout != -1) wrch('\n');
}

char printbuffer[32];

extern LispObject call1(const char *name, LispObject a1);
extern LispObject call2(const char *name, LispObject a1, LispObject a2);

void internalprint(LispObject x)
{   int sep = '(', i, esc, len;
    char *s;
    LispObject pn;
    switch (x & TAGBITS)
    {   case tagCONS:
            if (x == 0)    // can only occur in case of bugs here.
            {   wrch('#');
                return;
            }
            while (isCONS(x))
            {   i = printflags;
                if (qcar(x) == bignum &&
                    (pn = call1("~big2str", qcdr(x))) != NULLATOM &&
                    pn != nil)
                {   printflags = printPLAIN;
                    internalprint(pn);
                    printflags = i;
                    return;
                }
                printflags = i;
                checkspace(1);
                if (linepos != 0 || sep != ' ' || lispout < 0) wrch(sep);
                sep = ' ';
                push(x);
                internalprint(qcar(x));
                pop(x);
                x = qcdr(x);
            }
            if (x != nil)
            {   checkspace(3);
                wrch(' '); wrch('.'); wrch(' ');
                internalprint(x);
            }
            checkspace(1);
            wrch(')');
            return;
        case tagSYMBOL:
            pn = qpname(x);
            if (pn == nil)
            {   int len = sprintf(printbuffer, "g%.3d", gensymcounter++);
                push(x);
                pn = makestring(printbuffer, len);
                pop(x);
                qpname(x) = pn;
            }
            len = veclength(qheader(pn));
            s = qstring(pn);
            if ((printflags & printESCAPES) == 0)
            {   int i;
                checkspace(len);
                for (i=0; i<len; i++) wrch(s[i]);
            }
            else if (len != 0)
            {   esc = 0;
                if (!islower((int)s[0])) esc++;
                for (i=1; i<len; i++)
                {   if (!islower((int)s[i]) &&
                        !isdigit((int)s[i]) &&
                        s[i]!='_') esc++;
                }
                checkspace(len + esc);
                if (!islower((int)s[0])) wrch('!');
                wrch(s[0]);
                for (i=1; i<len; i++)
                {   if (!islower((int)s[i]) &&
                        !isdigit((int)s[i]) &&
                        s[i]!='_')
                        wrch('!');
                    wrch(s[i]);
                }
            }
            return;
        case tagATOM:
            if (x == NULLATOM)
            {   checkspace(5);
                wrch('#'); wrch('n'); wrch('u'); wrch('l'); wrch('l');
                return;
            }
            else switch (qheader(x) & TYPEBITS)
                {   case typeSTRING:
                        len = veclength(qheader(x));
                        s = qstring(x);
                        if ((printflags & printESCAPES) == 0)
                        {   int i;
                            checkspace(len);
                            for (i=0; i<len; i++) wrch(s[i]);
                        }
                        else
                        {   esc = 2;
                            for (i=0; i<len; i++)
                                if (s[i] == '"') esc++;
                            checkspace(len+esc);
                            wrch('"');
                            for (i=0; i<len; i++)
                            {   if (s[i] == '"') wrch('"');
                                wrch(s[i]);
                            }
                            wrch('"');
                        }
                        return;
                    case typeBIGNUM:
                        sprintf(printbuffer, "%" PRId64, qint64(x));
                        checkspace(len = strlen(printbuffer));
                        for (i=0; i<len; i++) wrch(printbuffer[i]);
                        return;
                    case typeVEC:
                    case typeEQHASH:
                    case typeEQHASHX:
                        sep = '[';
                        push(x);
                        for (i=0; i<veclength(qheader(TOS))/sizeof(LispObject); i++)
                        {   checkspace(1);
                            wrch(sep);
                            sep = ' ';
                            internalprint(elt(TOS, i));
                        }
                        pop(x);
                        checkspace(1);
                        wrch(']');
                        return;
                    default:
                        //case typeSYM:
                        // also the spare codes!
                        disaster(__LINE__);
                }
        case tagFLOAT:
            {   double d =  *((double *)(x - tagFLOAT));
                if (isnan(d)) strcpy(printbuffer, "NaN");
                else if (finite(d)) sprintf(printbuffer, "%.14g", d);
                else strcpy(printbuffer, "inf");
            }
            s = printbuffer;
// The C printing of floating point values is not to my taste, so I (slightly)
// asjust the output here...
            if (*s == '+' || *s == '-') s++;
            while (isdigit((int)*s)) s++;
            if (*s == 0 || *s == 'e')  // No decimal point present!
            {   len = strlen(s);
                while (len >= 0)       // Move existing text up 2 places
                {   s[len+2] = s[len];
                    len--;
                }
                s[0] = '.'; s[1] = '0'; // insert ".0"
            }
            checkspace(len = strlen(printbuffer));
            for (i=0; i<len; i++) wrch(printbuffer[i]);
            return;
        case tagFIXNUM:
            sprintf(printbuffer, "%" PRId64, (int64_t)qfixnum(x));
            checkspace(len = strlen(printbuffer));
            for (i=0; i<len; i++) wrch(printbuffer[i]);
            return;
        default:
//case tagFORWARD:
//case tagHDR:
            disaster(__LINE__);
    }
}

LispObject prin(LispObject a)
{   printflags = printESCAPES;
    push(a);
    internalprint(a);
    pop(a);
    return a;
}

LispObject princ(LispObject a)
{   printflags = printPLAIN;
    push(a);
    internalprint(a);
    pop(a);
    return a;
}

LispObject print(LispObject a)
{   printflags = printESCAPES;
    push(a);
    internalprint(a);
    pop(a);
    wrch('\n');
    return a;
}

void errprint(LispObject a)
{   int saveout = lispout, saveflags = printflags;
    lispout = 1; printflags = printESCAPES;
    internalprint(a);
    wrch('\n');
    lispout = saveout; printflags = saveflags;
}

LispObject printc(LispObject a)
{   printflags = printPLAIN;
    push(a);
    internalprint(a);
    pop(a);
    wrch('\n');
    return a;
}

int curchar = '\n', symtype = 0;

int hexval(int n)
{   if (isdigit(n)) return n - '0';
    else if ('a' <= n && n <= 'f') return n - 'a' + 10;
    else if ('A' <= n && n <= 'F') return n - 'A' + 10;
    else return 0;
}

LispObject token()
{   symtype = 'a';           // Default result is an atom.
    while (1)
    {   while (curchar == ' ' ||
               curchar == '\t' ||
               curchar == '\n') curchar = rdch(); // Skip whitespace
// Discard comments from "%" to end of line.
        if (curchar == '%')
        {   while (curchar != '\n' &&
                   curchar != EOF) curchar = rdch();
            continue;
        }
        break;
    }
    if (curchar == EOF)
    {   symtype = curchar;
        return NULLATOM;     // End of file marker.
    }
    if (curchar == '(' || curchar == '.' ||
        curchar == ')' || curchar == '\'' ||
        curchar == '`' || curchar == ',')
    {   symtype = curchar;   // Lisp special characters.
        curchar = rdch();
        if (symtype == ',' && curchar == '@')
        {   symtype = '@';
            curchar = rdch();
        }
        return NULLATOM;
    }
    boffop = 0;
    if (isalpha(curchar) || curchar == '!') // Start a symbol.
    {   while (isalpha(curchar) ||
               isdigit(curchar) ||
               curchar == '_' ||
               curchar == '!')
        {   if (curchar == '!') curchar = rdch();
            else if (curchar != EOF && qvalue(lower) != nil) curchar = tolower(curchar);
            else if (curchar != EOF && qvalue(raise) != nil) curchar = toupper(curchar);
            if (curchar != EOF)
            {   if (boffop < BOFFO_SIZE) boffo[boffop++] = curchar;
                curchar = rdch();
            }
        }
        boffo[boffop] = 0;
        return lookup(boffo, boffop, 1);
    }
    if (curchar == '"')                     // Start a string
    {   curchar = rdch();
        while (1)
        {   while (curchar != '"' && curchar != EOF)
            {   if (boffop < BOFFO_SIZE) boffo[boffop++] = curchar;
                curchar = rdch();
            }
// Note that a double-quote can be repeated within a string to denote
// a string with that character within it. As in
//   "abc""def"   is a string with contents   abc"def.
            if (curchar != EOF) curchar = rdch();
            if (curchar != '"') break;
            if (boffop < BOFFO_SIZE) boffo[boffop++] = curchar;
            curchar = rdch();
        }
        return makestring(boffo, boffop);
    }
    if (curchar == '+' || curchar == '-')
    {   boffo[boffop++] = curchar;
        curchar = rdch();
// + and - are treated specially, since if followed by a digit they
// introduce a (signed) number, but otherwise they are treated as punctuation.
        if (!isdigit(curchar))
        {   boffo[boffop] = 0;
            return lookup(boffo, boffop, 1);
        }
    }
// Note that in some cases after a + or - I drop through to here.
    if (curchar == '0' && boffop == 0)  // "0" without a sign in front
    {   boffo[boffop++] = curchar;
        curchar = rdch();
        if (curchar == 'x' || curchar == 'X') // Ahah - hexadecimal input
        {   LispObject r;
            boffop = 0;
            curchar = rdch();
            while (isxdigit(curchar))
            {   if (boffop < BOFFO_SIZE) boffo[boffop++] = curchar;
                curchar = rdch();
            }
            r = packfixnum(0);
            boffop = 0;
            while (boffo[boffop] != 0)
            {   r = call2("plus2", call2("times2", packfixnum(16), r),
                           packfixnum(hexval(boffo[boffop++])));
            }
            return r;
        }
    }
    if (isdigit(curchar) || (boffop == 1 && boffo[0] == '0'))
    {   while (isdigit(curchar))
        {   if (boffop < BOFFO_SIZE) boffo[boffop++] = curchar;
            curchar = rdch();
        }
// At this point I have a (possibly signed) integer. If it is immediately
// followed by a "." then a floating point value is indicated.
        if (curchar == '.')
        {   symtype = 'f';
            if (boffop < BOFFO_SIZE) boffo[boffop++] = curchar;
            curchar = rdch();
            while (isdigit(curchar))
            {   if (boffop < BOFFO_SIZE) boffo[boffop++] = curchar;
                curchar = rdch();
            }
// To make things tidy If I have a "." not followed by any digits I will
// insert a "0".
            if (!isdigit((int)boffo[boffop-1])) boffo[boffop++] = '0';
        }
// Whether or not there was a ".", an "e" or "E" introduces an exponent and
// hence indicates a floating point value.
        if (curchar == 'e' || curchar == 'E')
        {   symtype = 'f';
            if (boffop < BOFFO_SIZE) boffo[boffop++] = curchar;
            curchar = rdch();
            if (curchar == '+' || curchar == '-')
            {   if (boffop < BOFFO_SIZE) boffo[boffop++] = curchar;
                curchar = rdch();
            }
            while (isdigit(curchar))
            {   if (boffop < BOFFO_SIZE) boffo[boffop++] = curchar;
                curchar = rdch();
            }
// If there had been an "e" I force at least one digit in following it.
            if (!isdigit((int)boffo[boffop-1])) boffo[boffop++] = '0';
        }
        boffo[boffop] = 0;
        if (symtype == 'a')
        {   int neg = 0;
            LispObject r = packfixnum(0);
            boffop = 0;
            if (boffo[boffop] == '+') boffop++;
            else if (boffo[boffop] == '-') neg=1, boffop++;
            while (boffo[boffop] != 0)
            {   r = call2("plus2", call2("times2", packfixnum(10), r),
                           packfixnum(boffo[boffop++] - '0'));
            }
            if (neg) r = call1("minus", r);
            return r;
        }
        else
        {   double d;
            sscanf(boffo, "%lg", &d);
            return boxfloat(d);
        }
    }
    boffo[boffop++] = curchar;
    curchar = rdch();
    boffo[boffop] = 0;
    symtype = 'a';
    return lookup(boffo, boffop, 1);
}

// Syntax for Lisp input
//
//   S ::= name
//     |   integer
//     |   float
//     |   string
//     |   ' S   | ` S  | , S  | ,@ S
//     |   ( T
//     ;
//
//   T ::= )
//     |   . S )
//     |   S T
//     ;

extern LispObject readT();

LispObject readS()
{   LispObject q, w;
    while (1)
    {   switch (symtype)
        {   case '?':
                cursym = token();
                continue;
            case '(':
                cursym = token();
                return readT();
            case '.':
            case ')':     // Ignore spurious "." and ")" input.
                cursym = token();
                continue;
            case '\'':
                w = quote;
                break;
            case '`':
                w = backquote;
                break;
            case ',':
                w = comma;
                break;
            case '@':
                w = comma_at;
                break;
            case EOF:
                return eofsym;
            default:
                symtype = '?';
                return cursym;
        }
        push(w);
        cursym = token();
        q = readS();
        pop(w);
        return list2star(w, q, nil);
    }
}

LispObject readT()
{   LispObject q, r;
    if (symtype == '?') cursym = token();
    switch (symtype)
    {   case EOF:
            return eofsym;
        case '.':
            cursym = token();
            q = readS();
            if (symtype == '?') cursym = token();
            if (symtype == ')') symtype = '?'; // Ignore if not ")".
            return q;
        case ')':
            symtype = '?';
            return nil;
        // case '(':  case '\'':
        // case '`':  case ',':
        // case '@':
        default:
            q = readS();
            push(q);
            r = readT();
            pop(q);
            return cons(q, r);
    }
}

LispObject lookup(const char *s, int len, int createp)
{   LispObject w, pn;
    int i, hash = 1;
    for (i=0; i<len; i++) hash = 13*hash + s[i];
    hash = (hash & 0x7fffffff) % OBHASH_SIZE;
    w = obhash[hash];
    while (w != tagFIXNUM)
    {   LispObject a = qcar(w);        // Will be a symbol.
        LispObject n = qpname(a);      // Will be a string.
        int l = veclength(qheader(n)); // Length of the name.
        if (l == len &&
            strncmp(s, qstring(n), len) == 0)
            return a;                  // Existing symbol found.
        w = qcdr(w);
    }
// here the symbol as required was not already present.
    if (!createp) return undefined;
    pn = makestring(s, len);
    push(pn);
    w = allocatesymbol();
    pop(pn);
    qflags(w) = tagHDR + typeSYM;
    qvalue(w) = undefined;
    qplist(w) = nil;
    qpname(w) = pn;
    qdefn(w)  = NULL;
    qlits(w)  = nil;
    push(w);
    obhash[hash] = cons(w, obhash[hash]);
    pop(w);
    return w;
}

\end{verbatim}}
